\documentclass{article}

\usepackage{fancyhdr}
\usepackage{lastpage}
\usepackage{extramarks}
\usepackage[usenames,dvipsnames]{color}
\usepackage{amsmath}
\usepackage{amsthm}
\usepackage{amsfonts}
\usepackage{changepage}
\usepackage{lineno}
\usepackage[plain]{algorithm}
\usepackage{algpseudocode}
\usepackage{hyperref}
\usepackage{tikz}
\usepackage{listings}
\usepackage{graphics}
\usepackage{subcaption}
\usepackage{float}
\usepackage{fancyvrb}

\topmargin=-0.45in
\evensidemargin=0in
\oddsidemargin=0in
\textwidth=6.5in
\textheight=9.0in
\headsep=0.25in

\linespread{1.1}

\pagestyle{fancy}
\lhead{\hmwkAuthorName}
\chead{\hmwkClass\ (\hmwkClassInstructor\ \hmwkClassTime): \hmwkTitle}
\rhead{\firstxmark}
\lfoot{\lastxmark}
\cfoot{}

\renewcommand\headrulewidth{0.4pt}
\renewcommand\footrulewidth{0.4pt}

\setlength{\floatsep}{100pt}
\renewcommand{\algorithmicrequire}{\textbf{Input:}}
\renewcommand{\algorithmicensure}{\textbf{Output:}}
\algrenewcomment[1]{\hfill // #1}

\setlength\parindent{0pt}

\hypersetup{colorlinks=true}

\newcommand{\enterProblemHeader}[1]{
    \nobreak\extramarks{}{Problem \arabic{#1} continued on next page\ldots}\nobreak{}
    \nobreak\extramarks{Problem \arabic{#1} (continued)}{Problem \arabic{#1} continued on next page\ldots}\nobreak{}
}

\newcommand{\exitProblemHeader}[1]{
    \nobreak\extramarks{Problem \arabic{#1} (continued)}{Problem \arabic{#1} continued on next page\ldots}\nobreak{}
    \stepcounter{#1}
    \nobreak\extramarks{Problem \arabic{#1}}{}\nobreak{}
}

\setcounter{secnumdepth}{0}
\newcounter{partCounter}
\newcounter{homeworkProblemCounter}
\setcounter{homeworkProblemCounter}{1}
\nobreak\extramarks{Problem \arabic{homeworkProblemCounter}}{}\nobreak{}

\newenvironment{homeworkProblem}{
    \section{Problem \arabic{homeworkProblemCounter}}
    \setcounter{partCounter}{1}
    \enterProblemHeader{homeworkProblemCounter}
}{
    \exitProblemHeader{homeworkProblemCounter}
}

\newcommand{\hmwkTitle}{Homework\ \#2}
\newcommand{\hmwkDueDate}{February 28, 2014}
\newcommand{\hmwkClass}{ComS 573}
\newcommand{\hmwkClassTime}{10am}
\newcommand{\hmwkClassInstructor}{Professor De Brabanter}
\newcommand{\hmwkAuthorName}{Josh Davis}

\title{
    \vspace{2in}
    \textmd{\textbf{\hmwkClass:\ \hmwkTitle}}\\
    \normalsize\vspace{0.1in}\small{Due\ on\ \hmwkDueDate}\\
    \vspace{0.1in}\large{\textit{\hmwkClassInstructor\ at\ \hmwkClassTime}}
    \vspace{3in}
}

\author{\textbf{\hmwkAuthorName}}
\date{}

\newcommand{\alg}[1]{\textsc{\bfseries \footnotesize #1}}
\newcommand{\deriv}[1]{\frac{\mathrm{d}}{\mathrm{d}x} (#1)}
\newcommand{\pderiv}[2]{\frac{\partial}{\partial #1} (#2)}
\newcommand{\dx}{\mathrm{d}x}
\newcommand{\solution}{\textbf{\large Solution}}

\newcommand{\E}{\mathrm{E}}
\newcommand{\Var}{\mathrm{Var}}
\newcommand{\Cov}{\mathrm{Cov}}
\newcommand{\Bias}{\mathrm{Bias}}
\newcommand{\dist}[1]{\sim \mathrm{#1}}

\renewcommand{\part}[1]{\textbf{\large Part \Alph{partCounter}}\stepcounter{partCounter}\\}

\begin{document}

\maketitle

\pagebreak

\begin{homeworkProblem}
    Using a created simulated data, answer the questions regarding simple
    linear regression.
    \\

    \part

    Write out the form of the linear model. What are the regression
    coefficients?
    \\

    \solution

    Solution
    \\

    \part

    What is the correlation between \(x1\) and \(x2\)? Create a scatterplot
    displaying the relationship between the variables.
    \\

    \solution

    Solution.
    \\

    \part

    Using the data, fit a least squares regression to predict \(Y\) using
    \(x1\) and \(x2\). Describe the results obtained. What are
    \(\hat{\beta_0}\) and \(\hat{\beta_1}\), and \(\hat{\beta_2}\)? How do
    these relate to the true \(\beta_0\), \(\beta_1\), \(\beta_2\)? Can you
    reject the null hypothesis \(H_0 : \beta_1 = 0\)? How about \(H_0 : \beta_2
    = 0\)?
    \\

    \solution

    Solution.
    \\

    \part

    Now fit a least squares regression to predict \(Y\) using only \(x1\).
    Comment on your results. Can you reject the null hypothesis \(H_0 : \beta_1
    = 0\)?
    \\

    \solution

    Solution.
    \\\

    \part

    Now fit a least squares regression to predict \(Y\) using only \(x2\).
    Comment on your results. Can you reject the null hypothesis \(H_0 : \beta_1
    = 0\)?
    \\

    \part

    Do the results obtained in c-e contradict each other? Explain your answer.
    \\

    \solution

    Solution.
    \\

    \part

    Now suppose we obtain one additional observation, which was unfortunately
    mismeasured.

    \begin{verbatim}
> x1 <- c(x1, 0.1)
> x2 <- c(x2, 0.8)
> y <- c(y, 6)
    \end{verbatim}

    Re-fit the lienar models from c-e using this new data. What effects does
    this new observation have on each of the models? In each model, is this
    observation an outlier? A high leverage point?  Both? Explain your answers
    and make suitable plots.
    \\

    \solution

    Solution.
\end{homeworkProblem}

\pagebreak

\begin{homeworkProblem}
    This problem relates to the QDA model, in which the obesrvations within
    each class are drawn from a normal distribution with a class specific mean
    vector and a class specific covariance matrix. We consider the simple case
    where \(p = 1\), there is only one feature. Suppose that we have \(K\)
    classes, and if an observation belongs to the \(kth\) class then \(X\)
    comes from a one-dimensional normal distribution, \(X \dist{N}(\mu_k,
    \sigma_k^2)\).  Recall that the density function for the one-dimensional
    normal distribution is given in Eq. 4.11 in the text.  Prove that in this
    case, the Bayes classifier is not linear. Argue that it is in fact
    quadract.
    \\

    \solution

    Solution.
\end{homeworkProblem}

\end{document}
