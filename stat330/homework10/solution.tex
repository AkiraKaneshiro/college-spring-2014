\documentclass{article}

\usepackage{fancyhdr}
\usepackage{extramarks}
\usepackage{amsmath}
\usepackage{amsthm}
\usepackage{amsfonts}
\usepackage{multirow}
\usepackage{enumerate}

\topmargin=-0.45in
\evensidemargin=0in
\oddsidemargin=0in
\textwidth=6.5in
\textheight=9.0in
\headsep=0.25in

\linespread{1.1}

\pagestyle{fancy}
\lhead{\hmwkAuthorName}
\chead{\hmwkClass\ (\hmwkClassInstructor\ \hmwkClassTime): \hmwkTitle}
\rhead{\firstxmark}
\lfoot{\lastxmark}
\cfoot{\thepage}

\renewcommand\headrulewidth{0.4pt}
\renewcommand\footrulewidth{0.4pt}

\setlength\parindent{0pt}

\newcommand{\enterProblemHeader}[1]{
    \nobreak\extramarks{}{Problem \arabic{#1} continued on next page\ldots}\nobreak{}
    \nobreak\extramarks{Problem \arabic{#1} (continued)}{Problem \arabic{#1} continued on next page\ldots}\nobreak{}
}

\newcommand{\exitProblemHeader}[1]{
    \nobreak\extramarks{Problem \arabic{#1} (continued)}{Problem \arabic{#1} continued on next page\ldots}\nobreak{}
    \stepcounter{#1}
    \nobreak\extramarks{Problem \arabic{#1}}{}\nobreak{}
}

\setcounter{secnumdepth}{0}
\newcounter{partCounter}
\newcounter{homeworkProblemCounter}
\setcounter{homeworkProblemCounter}{1}
\nobreak\extramarks{Problem \arabic{homeworkProblemCounter}}{}\nobreak{}

\newenvironment{homeworkProblem}{
    \section{Problem \arabic{homeworkProblemCounter}}
    \setcounter{partCounter}{1}
    \enterProblemHeader{homeworkProblemCounter}
}{
    \exitProblemHeader{homeworkProblemCounter}
}

\newcommand{\hmwkTitle}{Homework\ \#10}
\newcommand{\hmwkDueDate}{April 23, 2014}
\newcommand{\hmwkClass}{Stat330}
\newcommand{\hmwkClassTime}{Section A}
\newcommand{\hmwkClassInstructor}{Mr. Lanker}
\newcommand{\hmwkAuthorName}{Josh Davis}

\title{
    \vspace{2in}
    \textmd{\textbf{\hmwkClass:\ \hmwkTitle}}\\
    \normalsize\vspace{0.1in}\small{Due\ on\ \hmwkDueDate\ at 3:10pm}\\
    \vspace{0.1in}\large{\textit{\hmwkClassInstructor\ \hmwkClassTime}}
    \vspace{3in}
}

\author{\textbf{\hmwkAuthorName}}
\date{}

\newcommand{\deriv}[1]{\frac{\mathrm{d}}{\mathrm{d}x} (#1)}
\newcommand{\pderiv}[2]{\frac{\partial}{\partial #1} (#2)}
\newcommand{\dx}{\mathrm{d}x}
\newcommand{\solution}{\textbf{\large Solution}}

\newcommand{\E}{\mathrm{E}}
\newcommand{\Var}{\mathrm{Var}}
\newcommand{\Cov}{\mathrm{Cov}}
\newcommand{\Bias}{\mathrm{Bias}}
\newcommand{\Std}{\mathrm{Std}}
\newcommand{\dist}[1]{\sim \mathrm{#1}}
\newcommand{\pval}{\(p\)-value}
\newcommand{\tstat}{\(t\)-statistic}
\newcommand{\Likelihood}{\mathcal{L}}

\renewcommand{\part}[1]{\textbf{\large Part \Alph{partCounter}}\stepcounter{partCounter}\\}

\begin{document}

\maketitle

\pagebreak

\begin{homeworkProblem}
    Let \(\bar{X}\) denote the mean of a random sample of \(n\) i.i.d.
    observations from a distribution that is \emph{Normal}\((\mu, \sigma^2)\),
    where \(\sigma^2 > 0\), and \(\sigma\) is known but \(\mu\) is unknown.
    \\

    What is the probability that the confidence interval:
    \[
        (
            \bar{X}
            - 2.2
            \frac{
                \sigma
            }{
                \sqrt{n}
            },
            \bar{X}
            + 2.2
            \frac{
                \sigma
            }{
                \sqrt{n}
            }
        )
    \]

    contains the fixed point \(\mu\)?
    \\

    \solution

    Solution.
\end{homeworkProblem}

\pagebreak

\begin{homeworkProblem}
    Problem 9.7a from Baron on pg. 301.
    \\

    \solution

    Solution.
\end{homeworkProblem}

\pagebreak

\begin{homeworkProblem}
    Find a 95\% confidence interval for \(\mu\), the true mean of a normal
    population which has a variance of \(\sigma^2 = 100\).  Consider a smaple
    of size 25 that has a mean of 69.3.
    \\

    \solution

    Solution.
\end{homeworkProblem}

\pagebreak

\begin{homeworkProblem}
    A department store has 10,000 customers charge accounts. To establish the
    amount owed by all its customers, it selected 36 accounts at random and
    found a mean of \$150 and a standard deviation of \$60.
    \\

    \part

    Establish a 95\% confidence interval estimate of the mean amount owed by
    its customers.
    \\

    \solution

    Solution.
    \\

    \part

    Provide an interpretation for this confidence interval to someone with
    little statistical background. \emph{Hint:} See pg. 248.
    \\

    \solution

    Solution.
\end{homeworkProblem}

\pagebreak

\begin{homeworkProblem}
    Find a 90\% confidence interval for \(\mu_1 - \mu_2\) when \(n_1 = 30\),
    \(n_2 = 39\), \(\bar{x}_1 = 4.2\), \(\bar{x}_2 = 3.4\), \(s^2_1 = 49\) and
    \(s^2_2 = 32\).
    \\

    \solution

    Solution.
\end{homeworkProblem}

\pagebreak

\begin{homeworkProblem}
    Problem 9.9a from Baron on pg. 301.
    \\

    \solution

    Solution.
\end{homeworkProblem}

\pagebreak

\begin{homeworkProblem}
    Cranston, Rhode Island, has the reputation for selling the most expensive
    bubble gum in the U.S.  Ten candy stores were surveyed and it was found
    that the average price in the 10 stores was 40 cents with a standard
    deviation of 5 cents.
    \\

    \part

    Find a 95\% confidence interval for \(\mu\), the mean gum price.
    \\

    \solution

    Solution.
    \\

    \part

    Find a 99\% confidence interval for \(\mu\), the mean gum price.
    \\

    \solution

    Solution.
\end{homeworkProblem}

\end{document}
