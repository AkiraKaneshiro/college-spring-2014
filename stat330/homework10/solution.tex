\documentclass{article}

\usepackage{fancyhdr}
\usepackage{extramarks}
\usepackage{amsmath}
\usepackage{amsthm}
\usepackage{amsfonts}
\usepackage{multirow}
\usepackage{enumerate}

\topmargin=-0.45in
\evensidemargin=0in
\oddsidemargin=0in
\textwidth=6.5in
\textheight=9.0in
\headsep=0.25in

\linespread{1.1}

\pagestyle{fancy}
\lhead{\hmwkAuthorName}
\chead{\hmwkClass\ (\hmwkClassInstructor\ \hmwkClassTime): \hmwkTitle}
\rhead{\firstxmark}
\lfoot{\lastxmark}
\cfoot{\thepage}

\renewcommand\headrulewidth{0.4pt}
\renewcommand\footrulewidth{0.4pt}

\setlength\parindent{0pt}

\newcommand{\enterProblemHeader}[1]{
    \nobreak\extramarks{}{Problem \arabic{#1} continued on next page\ldots}\nobreak{}
    \nobreak\extramarks{Problem \arabic{#1} (continued)}{Problem \arabic{#1} continued on next page\ldots}\nobreak{}
}

\newcommand{\exitProblemHeader}[1]{
    \nobreak\extramarks{Problem \arabic{#1} (continued)}{Problem \arabic{#1} continued on next page\ldots}\nobreak{}
    \stepcounter{#1}
    \nobreak\extramarks{Problem \arabic{#1}}{}\nobreak{}
}

\setcounter{secnumdepth}{0}
\newcounter{partCounter}
\newcounter{homeworkProblemCounter}
\setcounter{homeworkProblemCounter}{1}
\nobreak\extramarks{Problem \arabic{homeworkProblemCounter}}{}\nobreak{}

\newenvironment{homeworkProblem}{
    \section{Problem \arabic{homeworkProblemCounter}}
    \setcounter{partCounter}{1}
    \enterProblemHeader{homeworkProblemCounter}
}{
    \exitProblemHeader{homeworkProblemCounter}
}

\newcommand{\hmwkTitle}{Homework\ \#10}
\newcommand{\hmwkDueDate}{April 23, 2014}
\newcommand{\hmwkClass}{Stat330}
\newcommand{\hmwkClassTime}{Section A}
\newcommand{\hmwkClassInstructor}{Mr. Lanker}
\newcommand{\hmwkAuthorName}{Josh Davis}

\title{
    \vspace{2in}
    \textmd{\textbf{\hmwkClass:\ \hmwkTitle}}\\
    \normalsize\vspace{0.1in}\small{Due\ on\ \hmwkDueDate\ at 3:10pm}\\
    \vspace{0.1in}\large{\textit{\hmwkClassInstructor\ \hmwkClassTime}}
    \vspace{3in}
}

\author{\textbf{\hmwkAuthorName}}
\date{}

\newcommand{\deriv}[1]{\frac{\mathrm{d}}{\mathrm{d}x} (#1)}
\newcommand{\pderiv}[2]{\frac{\partial}{\partial #1} (#2)}
\newcommand{\dx}{\mathrm{d}x}
\newcommand{\solution}{\textbf{\large Solution}}

\newcommand{\E}{\mathrm{E}}
\newcommand{\Var}{\mathrm{Var}}
\newcommand{\Cov}{\mathrm{Cov}}
\newcommand{\Bias}{\mathrm{Bias}}
\newcommand{\Std}{\mathrm{Std}}
\newcommand{\dist}[1]{\sim \mathrm{#1}}
\newcommand{\pval}{\(p\)-value}
\newcommand{\tstat}{\(t\)-statistic}
\newcommand{\tdist}{\(t\)-distribution}
\newcommand{\Likelihood}{\mathcal{L}}

\renewcommand{\part}[1]{\textbf{\large Part \Alph{partCounter}}\stepcounter{partCounter}\\}

\begin{document}

\maketitle

\pagebreak

\begin{homeworkProblem}
    Let \(\bar{X}\) denote the mean of a random sample of \(n\) i.i.d.
    observations from a distribution that is \emph{Normal}\((\mu, \sigma^2)\),
    where \(\sigma^2 > 0\), and \(\sigma\) is known but \(\mu\) is unknown.
    \\

    What is the probability that the confidence interval:
    \[
        (
            \bar{X}
            - 2.2
            \frac{
                \sigma
            }{
                \sqrt{n}
            },
            \bar{X}
            + 2.2
            \frac{
                \sigma
            }{
                \sqrt{n}
            }
        )
    \]

    contains the fixed point \(\mu\)?
    \\

    \solution

    According to the standard Z table, when \(z_{\alpha / 2} = 2.2\), the
    probability is 0.9861.
\end{homeworkProblem}

\begin{homeworkProblem}
    Problem 9.7a from Baron on pg. 301. Answer in the back of the book.
    \\

    Let there be a server that serves concurrent users. Let the average number
    of concurrent users at 100 randomly selected times be 37.7 with a standard
    deviation of \(\sigma = 9.2\).
    \\

    Construct a 90\% confidence interval for the expectation of the number of
    concurrent users.
    \\

    \solution

    We use the standard equation for confidence interval when the mean when
    \(\sigma\) is known. We know that \(\bar{X} = 37.7\), \(n = 100\), and
    \(\alpha = 0.1\). This gives us a \(z_{\alpha / 2} = z_{0.05} = 1.645\) by
    using a table.  This gives us:
    \[
        \bar{X} \pm z_{\alpha / 2}
        \frac{
            \sigma
        }{
            \sqrt{n}
        }
        =
        37.7
        \pm 1.645
        \frac{
            9.2
        }{
            10
        }
        =
        37.7
        \pm
        1.513
        =
        \left[
            36.19,
            39.21
        \right]
    \]
\end{homeworkProblem}

\begin{homeworkProblem}
    Find a 95\% confidence interval for \(\mu\), the true mean of a normal
    population which has a variance of \(\sigma^2 = 100\).  Consider a sample
    of size 25 that has a mean of 69.3.
    \\

    \solution

    We know that \(\bar{X} = 69.3\), \(n = 25\), and \(\alpha = 0.05\). This
    gives us a \(z_{\alpha / 2} = z_{0.025} = 1.96\) by using a table.  This
    gives us:
    \[
        \bar{X} \pm z_{\alpha / 2}
        \frac{
            \sigma
        }{
            \sqrt{n}
        }
        =
        69.3 \pm 1.96
        \frac{
            10
        }{
            5
        }
        =
        69.3
        \pm
        3.92
        =
        \left[
            65.08,
            72.92
        \right]
    \]
\end{homeworkProblem}

\pagebreak

\begin{homeworkProblem}
    A department store has 10,000 customers charge accounts. To establish the
    amount owed by all its customers, it selected 36 accounts at random and
    found a mean of \$150 and a standard deviation of \$60.
    \\

    \part

    Establish a 95\% confidence interval estimate of the mean amount owed by
    its customers.
    \\

    \solution

    We know that \(\bar{X} = 150\), \(n = 36\), and \(\alpha = 0.05\). This
    gives us a \(z_{\alpha / 2} = z_{0.025} = 1.96\) by using a table. This
    gives us:
    \[
        \bar{X} \pm z_{\alpha / 2}
        \frac{
            \sigma
        }{
            \sqrt{n}
        }
        =
        150 \pm 1.96
        \frac{
            60
        }{
            6
        }
        =
        150
        \pm
        19.6
        =
        \left[
            130.4,
            169.6
        \right]
    \]

    \part

    Provide an interpretation for this confidence interval to someone with
    little statistical background. \emph{Hint:} See pg. 248.
    \\

    \solution

    Say there is a piece of paper in a sealed envelope with a number from 1 to
    200 on it. Then let's say that we don't know what this number is. By saying
    that our 95\% confidence interval starts at 130.4 and goes to 169.6, we are
    saying that if we open the envelope and look at the number, we have a 95\%
    chance that our range covers the number on the piece of paper.
\end{homeworkProblem}

\begin{homeworkProblem}
    Find a 90\% confidence interval for \(\mu_1 - \mu_2\) when \(n_1 = 30\),
    \(n_2 = 39\), \(\bar{x}_1 = 4.2\), \(\bar{x}_2 = 3.4\), \(s^2_1 = 49\) and
    \(s^2_2 = 32\).
    \\

    \solution

    This is a confidence interval for differene of means. This gives us \(\alpha = 0.1\),
    so \(z_{\alpha / 2} = z_{0.05} = 1.645\) then:
    \[
        \bar{x}_1 - \bar{x}_2
        \pm z_{\alpha / 2}
        \sqrt{
            \frac{
                s^2_1
            }{
                n_1
            }
            +
            \frac{
                s^2_2
            }{
                n_2
            }
        }
        =
        4.2 - 3.4
        \pm 1.645
        \sqrt{
            \frac{
                49
            }{
                30
            }
            +
            \frac{
                32
            }{
                39
            }
        }
        =
        0.8
        \pm
        2.58
        =
        \left[
            -1.78,
            3.38
        \right]
    \]

\end{homeworkProblem}

\begin{homeworkProblem}
    Problem 9.9a from Baron on pg. 301. Answer in the back of the book.
    \\

    Salaries of entry level computer engineers have a Normal distribution with
    unknown mean and variance. Three randomly selected computer engineers have
    salaries: 30, 50, 70.
    \\

    Construct a 90\% confidence interval for the average salary of an
    entry-level computer engineer.
    \\

    \solution

    We have a small sample size, so we need to use the {\tstat}. Using our
    sample, this gives us: \(\bar{X} = 50\), \(\alpha = 0.10\), \(s = 20\).
    Using a \(t\) table, know our {\tstat} has 2 degrees of freedom which gives
    \(t_{0.05} = 2.353\). Thus:
    \[
        \bar{X} \pm t_{\alpha / 2}
        \frac{
            s
        }{
            \sqrt{n}
        }
        =
        50 \pm 2.920
        \frac{
            20
        }{
            \sqrt{3}
        }
        =
        50
        \pm
        33.72
        =
        \left[
            16.3,
            83.7
        \right]
    \]
\end{homeworkProblem}

\pagebreak

\begin{homeworkProblem}
    Cranston, Rhode Island, has the reputation for selling the most expensive
    bubble gum in the U.S.  Ten candy stores were surveyed and it was found
    that the average price in the 10 stores was 40 cents with a standard
    deviation of 5 cents.
    \\

    \part

    Find a 95\% confidence interval for \(\mu\), the mean gum price.
    \\

    \solution

    With our sample, this gives us: \(\bar{X} = 40\), \(\alpha = 0.05\), \(s = 5\).
    Using a \(t\) table, know our {\tstat} has 9 degrees of freedom which gives
    \(t_{0.025} = 2.262\). Thus:
    \[
        \bar{X} \pm t_{\alpha / 2}
        \frac{
            s
        }{
            \sqrt{n}
        }
        =
        40 \pm 2.262
        \frac{
            5
        }{
            \sqrt{10}
        }
        =
        40
        \pm
        3.58
        =
        \left[
            36.4,
            43.6
        \right]
    \]

    \part

    Find a 99\% confidence interval for \(\mu\), the mean gum price.
    \\

    \solution

    Our new value for \(\alpha\) is 0.01.  Using a \(t\) table, know our
    {\tstat} has 9 degrees of freedom which gives \(t_{0.005} = 3.250\). Thus:
    \[
        \bar{X} \pm t_{\alpha / 2}
        \frac{
            s
        }{
            \sqrt{n}
        }
        =
        40 \pm 3.250
        \frac{
            5
        }{
            \sqrt{10}
        }
        =
        40
        \pm
        5.14
        =
        \left[
            34.9,
            45.1
        \right]
    \]
\end{homeworkProblem}

\pagebreak

\begin{homeworkProblem}
\end{homeworkProblem}

\pagebreak

\begin{homeworkProblem}
    The mean yield of corn in the US is about 120 bushels per acre (from 1989).
    A survey of 50 farmers this year gives a sample mean yield of \(\bar{x} =
    123.6\) bushels per acre.  We want to know whether this is good evidence
    that the national mean this year is not 120 bushels per acre.  Assume that
    the sample is i.i.d. from the entire population and that the standard
    deviation of the yield in this population is \(\sigma = 10\) bushels per
    acre.
    \\

    Give the {\pval} for the test of
    \[
        H_0 : \mu = 120
        \quad
        vs.
        \quad
        H_a : \mu \neq 120
    \]

    Are you convinced that the population mean is not 120 bushels per acre? Use
    the 0.05 significance level in making your decision.
    \\

    \solution

    Solution.
\end{homeworkProblem}

\pagebreak

\begin{homeworkProblem}
    In the past, the mean score of the seniors at South High on the ACT exam
    has been 20.0.  This year a special preparation course is offered, and all
    43 seniors planning to take the ACT enroll in the course. The mean of their
    ACT scores is 21.1. Assume that the ACT scores vary normally with \(\sigma
    = 6\).
    \\

    Is the outcome good evidence that this class's true mean is not 20? State
    your hypotheses, compute the {\pval}, and assess the amount of evidence.
    \\

    \solution

    Solution.
\end{homeworkProblem}

\pagebreak

\begin{homeworkProblem}
\end{homeworkProblem}


\end{document}
