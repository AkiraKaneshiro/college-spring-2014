\documentclass{article}

\usepackage{fancyhdr}
\usepackage{lastpage}
\usepackage{extramarks}
\usepackage[usenames,dvipsnames]{color}
\usepackage{amsmath}
\usepackage{amsthm}
\usepackage{amsfonts}
\usepackage{changepage}
\usepackage{lineno}
\usepackage[plain]{algorithm}
\usepackage{algpseudocode}
\usepackage{hyperref}
\usepackage{tikz}

\topmargin=-0.45in
\evensidemargin=0in
\oddsidemargin=0in
\textwidth=6.5in
\textheight=9.0in
\headsep=0.25in

\linespread{1.1}

\pagestyle{fancy}
\lhead{\hmwkAuthorName}
\chead{\hmwkClass\ (\hmwkClassInstructor\ \hmwkClassTime): \hmwkTitle}
\rhead{\firstxmark}
\lfoot{\lastxmark}
\cfoot{}

\renewcommand\headrulewidth{0.4pt}
\renewcommand\footrulewidth{0.4pt}

\setlength{\floatsep}{100pt}
\renewcommand{\algorithmicrequire}{\textbf{Input:}}
\renewcommand{\algorithmicensure}{\textbf{Output:}}
\algrenewcomment[1]{\hfill // #1}

\setlength\parindent{0pt}

\hypersetup{colorlinks=true}

\newcommand{\enterProblemHeader}[1]{
    \nobreak\extramarks{}{Problem \arabic{#1} continued on next page\ldots}\nobreak{}
    \nobreak\extramarks{Problem \arabic{#1} (continued)}{Problem \arabic{#1} continued on next page\ldots}\nobreak{}
}

\newcommand{\exitProblemHeader}[1]{
    \nobreak\extramarks{Problem \arabic{#1} (continued)}{Problem \arabic{#1} continued on next page\ldots}\nobreak{}
    \stepcounter{#1}
    \nobreak\extramarks{Problem \arabic{#1}}{}\nobreak{}
}

\setcounter{secnumdepth}{0}
\newcounter{partCounter}
\newcounter{homeworkProblemCounter}
\setcounter{homeworkProblemCounter}{1}
\nobreak\extramarks{Problem \arabic{homeworkProblemCounter}}{}\nobreak{}

\newenvironment{homeworkProblem}{
    \section{Problem \arabic{homeworkProblemCounter}}
    \setcounter{partCounter}{1}
    \enterProblemHeader{homeworkProblemCounter}
}{
    \exitProblemHeader{homeworkProblemCounter}
}

\newcommand{\hmwkTitle}{Homework\ \#2}
\newcommand{\hmwkDueDate}{January 29, 2014}
\newcommand{\hmwkClass}{Stat330}
\newcommand{\hmwkClassTime}{Section A}
\newcommand{\hmwkClassInstructor}{Mr. Lanker}
\newcommand{\hmwkAuthorName}{Josh Davis}

\title{
    \vspace{2in}
    \textmd{\textbf{\hmwkClass:\ \hmwkTitle}}\\
    \normalsize\vspace{0.1in}\small{Due\ on\ \hmwkDueDate at 3:10pm}\\
    \vspace{0.1in}\large{\textit{\hmwkClassInstructor\ \hmwkClassTime}}
    \vspace{3in}
}

\author{\textbf{\hmwkAuthorName}}
\date{}

\newcommand{\alg}[1]{\textsc{\bfseries \footnotesize #1}}
\newcommand{\deriv}[1]{\frac{\mathrm{d}}{\mathrm{d}x} (#1)}
\newcommand{\dx}{\mathrm{d}x}
\newcommand{\solution}{\textbf{\large Solution}}

\renewcommand{\part}[1]{\textbf{\large Part \Alph{partCounter}}\stepcounter{partCounter}\\}


\begin{document}

\maketitle

\pagebreak


\begin{homeworkProblem}
    Binomial theorem.
    \\

    \part

    What is the coefficient of \(x^5 y^3\)?
    \\

    \solution

    \[
        c \cdot x^5 y^3 = \binom{8}{3} x^5 y^3= \frac{8!}{5!3!} x^5 y^3 = 56 x^5 y^3
    \]

    \part

    What is the coefficient of \(x^3 y^5\)?
    \\

    \solution

    \[
        c \cdot x^3 y^5 = \binom{8}{5} x^3 y^5= \frac{8!}{5!3!} x^3 y^5 = 56 x^3 y^5
    \]
\end{homeworkProblem}

\pagebreak

\begin{homeworkProblem}
    Simpson's Paradox.
    \\

    \part

    A black urn contains 5 red and 6 green balls and a white urn contains 3 red
    and 4 green balls. You are allowed to choose an urn and then choose a ball
    at random from the urn. Assume that choosing any ball in the urn is equally
    likely. If you choose a red ball you ge a prize. Which urn should you
    choose to draw from?
    \\

    \solution

    For the black urn, the probability of getting a red is:

    \[
        \Pr(\mbox{black})
        = \frac{5 \mbox{ red}}{5 + 6 \mbox{ total}}
        = \frac{5}{11} \approx .455
    \]

    \[
        \Pr(\mbox{white})
        = \frac{3 \mbox{ red}}{3 + 4 \mbox{ total}}
        = \frac{3}{7} \approx .429
    \]

    Therefore we have a higher probability of getting a red from the black urn
    and should choose from that one.
    \\

    \part

    A second black urn contains 6 red and 3 green balls and a second white urn
    contains 9 red and 5 green balls.
    \\

    \solution

    \[
        \Pr(\mbox{2nd black})
        = \frac{6 \mbox{ red}}{6 + 3 \mbox{ total}}
        = \frac{6}{9} \approx .667
    \]

    \[
        \Pr(\mbox{2nd white})
        = \frac{9 \mbox{ red}}{9 + 5 \mbox{ total}}
        = \frac{9}{14} \approx .643
    \]

    The 2nd black has a higher probability than the 2nd white so we should
    choose from that one.
    \\

    \part

    The two black urns are combined as well as the two white urns.
    \\

    \solution

    \[
        \Pr(\mbox{combined black})
        = \frac{5 + 6 \mbox{ red}}{11 + 9 \mbox{ total}}
        = \frac{11}{20} \approx .55
    \]

    \[
        \Pr(\mbox{combined white})
        = \frac{3 + 9 \mbox{ red}}{7 + 14 \mbox{ total}}
        = \frac{12}{21} \approx .571
    \]

    The combined white urn has a higher probability of getting a red so we
    should choose that one.
\end{homeworkProblem}

\pagebreak

\begin{homeworkProblem}
    A group of 4 undergraduates and 5 graduates are available to fill four
    student government posts.
    \\

    \part

    Find the probability that at least three undergraduates will be among the
    four chosen assuming the selection was random?
    \\

    \solution

    \[
        \frac{\mbox{at least 3 undergrads}}{\mbox{total possibilities}}
        = \frac{\mbox{exactly 3 } + \mbox{ 4 undergrads}}{\mbox{total possibilities}}
        = \frac{\binom{4}{3} \binom{5}{1} + \binom{4}{4} \binom{5}{0}}{\binom{9}{5}}
        = \frac{4 \cdot 5 + 1 \cdot 1}{126}
        = \frac{21}{126}
        \approx .167
    \]

    \part

    What is the probability of no undergraduates being selected again assuming
    the selection was random?
    \\

    \solution

    \[
        \frac{\mbox{no undergrads}}{\mbox{total possibilities}}
        = \frac{\binom{4}{0} \binom{5}{4}}{\binom{9}{5}}
        = \frac{1 \cdot 5}{126}
        \approx .040
    \]

    \part

    Later we will talk about a \(p\)-value, which represents the probability
    happening under certain model assumptions. Here our model assumptions are
    that the selection process was purely random. It is common that if the
    \(p\)-value is very low, say less than .05, we reject that our model
    assumptions are true. Our \(p\)-value here is the result from part b. What
    does this value say about the model statement that the selections were made
    at random?
    \\

    \solution

    Since our result from part b was \(\approx .040\) and since that is lower
    than .05, this says that our assumption that the selection process was
    purely random was false.
\end{homeworkProblem}

\begin{homeworkProblem}
    Suppose that there are two events \(A\) and \(B\) in \(\Omega\) such that
    \(\Pr(A) > 0\) and \(\Pr(B) > 0\).  Further, suppose that the two events are
    mutually exclusive. Can they also be independent? Explain.
    \\

    \solution

    We know that \(\Pr(A) > 0\), \(\Pr(B) > 0\), and that \(A\) and \(B\) are
    mutually exclusive which means that \(A \cap B = \{\}\). So that probability
    of \(\Pr(A \cap B) = 0\) is by definition.
    \\

    The definition of independent events means that \(\Pr(A \cap B) = \Pr(A)
    \Pr(B)\). This cannot be true because we know that \(\Pr(A \cap B) = 0\)
    and thus one of the probabilities of the events must be zero, which goes
    against the assumption. So no, they cannot also be independent.
\end{homeworkProblem}

\pagebreak

\begin{homeworkProblem}
    Let \(A\) be the event that processor 1 is in use and \(B\) be the event
    that processor 2 is in use.
    \\

    \part

    Calculate \(\Pr(A \mid B)\).
    \\

    \solution

    \[
        \Pr(A \mid B)
        = \frac{\Pr(A \cap B)}{\Pr(B)}
        = \frac{.7}{.8}
        = .875
    \]

    \part

    What is \(\Pr(A)\)?
    \\

    \solution

    \[
        \Pr(A)
        = .75
    \]

    \part

    Are \(A\) and \(B\) independent?
    \\

    \solution

    No, \(A\) and \(B\) are not independent because \(\Pr(A \mid B) \ne
    \Pr(A)\) as our previous parts show.
    \\

    \part

    Calculate \(\Pr(B \mid A)\).
    \\

    \solution

    \[
        \Pr(B \mid A)
        = \frac{\Pr(A \cap B)}{\Pr(A)}
        = \frac{.7}{.75}
        \approx .933
    \]

    \part

    Show that \(\Pr(A \mid B) \Pr(B) = \Pr(B \mid A) \Pr(A)\).
    \\

    \solution

    \begin{proof}
        \[
            \begin{split}
                \Pr(A \mid B) \Pr(B)
                &= \frac{\Pr(A \cap B)}{\Pr(B)} \Pr(B)
                \\
                &= \Pr(A \cap B)
                \\
                &= \frac{\Pr(A \cap B)}{\Pr(A)} \Pr(A)
                \\
                &= \Pr(B \mid A) \Pr(A)
            \end{split}
        \]
    \end{proof}

    Using our values:

    \[
        \begin{split}
            \Pr(A \mid B) \Pr(B)
            = (.875)(.8)
            \approx .700
            \\
            \Pr(B \mid A) \Pr(A)
            = (.933)(.75)
            \approx .700
        \end{split}
    \]
\end{homeworkProblem}

\pagebreak

\begin{homeworkProblem}
    Suppose that \(A\) and \(B\) are independent events with \(\Pr(A) > 0\) and
    \(\Pr(B) > 0\). Show that \(\overline{A}\) and \(\overline{B}\) are also
    independent.
    \\

    \solution

    \begin{proof}
        Suppose that \(A\) and \(B\) are events with probability \(\Pr(A) > 0\) and
        \(\Pr(B) > 0\). And that the events are also independent, which means
        that \(\Pr(A \cap B) = \Pr(A) \Pr(B)\).
        \\

        We want to show that \(\overline{A}\) and \(B\) are independent or in other
        words, we want to show that \(\Pr(\overline{A} \cap B) = \Pr(\overline{A})
        \Pr(B)\).
        \\

        Since \(B = (A \cap B) \cup (\overline{A} \cap B)\), then \(\Pr(B) = \Pr(A
        \cap B) + \Pr(\overline{A} \cap B)\). Which gives us:

        \[
            \begin{split}
                \Pr(\overline{A} \cap B) &= \Pr(B) - \Pr(A \cap B)
                \\
                &= \Pr(B) - \Pr(A) \Pr(B) \mbox{ because of independence}
                \\
                &= \Pr(B) (1 - \Pr(A))
                \\
                &= \Pr(B) \Pr(\overline{A})
            \end{split}
        \]

        Thus by assuming that \(A\) and \(B\) are independent, we have shown that
        \(\overline{A}\) and \(B\) are also independent, or \(\Pr(\overline{A} \cap
        B) = \Pr(\overline{A}) \Pr(B)\). Thus the proof is complete.
    \end{proof}
\end{homeworkProblem}

\begin{homeworkProblem}
    Firing based on positive drug use. Was she fired unjustly?
    \\

    Let \(A\) be the event that the person is a drug user and let \(B\) be the
    event that the person tested positive for drugs. We have the following
    equations:

    \[
            \Pr(A) = .02,
            \quad
            \Pr(\overline{A}) = .98,
            \quad
            \Pr(B \mid A) = .99,
            \quad
            \Pr(B \mid \overline{A}) = .01
    \]

    \solution

    We want to determine what the probability that she was not a drug user
    given that she tested positive, or \(\Pr(\overline{A} \mid B)\). By using
    Bayes Rule and the law of total probability we get:

    \[
            \begin{split}
                \Pr(\overline{A} \mid B) &= \frac{\Pr(B \mid \overline{A}) \Pr(\overline{A})}{\Pr(B)}
                \\
                &= \frac{
                    \Pr(B \mid \overline{A}) \Pr(\overline{A})
                }{
                    \Pr(B \mid \overline{A}) \Pr(\overline{A}) + \Pr(B \mid A) \Pr(A)
                }
                \\
                &= \frac{
                    (.01) (.98)
                }{
                    (.01) (.98) + (.99) (.02)
                }
                \\
                &\approx 33.1\%
            \end{split}
    \]

    Considering the possibility that the test is accurate is 99\% and the
    possibility that she is not a drug user given she tests positive (a false
    positive) is 4\%, I'd say she doesn't have a legitimate claim.
\end{homeworkProblem}

\pagebreak

\begin{homeworkProblem}
    In a bag there are two standard dice. One of the dice is fair and the other
    is loaded so that the die rolls a 6 exactly half of the time. You reach
    into the bag, randomly select a dice, and roll a 6. What is the probability
    that you selected the loaded die?
    \\

    \solution

    Let \(A\) be the event that we pick the loaded die and let \(B\) be the
    event that we roll a 6.
    \\

    Since there are two dice in the bag, \(\Pr(A) = .5\) because we are drawing
    at random. We also know that \(\Pr(B \mid A) = .5\) and \(\Pr(B \mid
    \overline{A}) = .167\) based on the problem description.
    \\

    We want to find the probability that we picked the loaded die given
    that we rolled a six which is \(\Pr(A \mid B)\).
    \\

    Using Bayes Rule and the law of total probability:

    \[
        \begin{split}
            \Pr(A \mid B) &= \frac{\Pr(B \mid A) \Pr(A)}{\Pr(B)}
            \\
            &= \frac{
                \Pr(B \mid A) \Pr(A)
            }{
                \Pr(B \mid A) \Pr(A) + \Pr(B \mid \overline{A}) \Pr(\overline{A})
            }
            \\
            &= \frac{
                (.5) (.5)
            }{
                (.5) (.5) + (.167) (1 - .5)
            }
            \\
            &\approx .750
        \end{split}
    \]

\end{homeworkProblem}

\end{document}
