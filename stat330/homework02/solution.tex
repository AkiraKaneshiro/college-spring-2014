\documentclass{article}

\usepackage{fancyhdr}
\usepackage{lastpage}
\usepackage{extramarks}
\usepackage[usenames,dvipsnames]{color}
\usepackage{amsmath}
\usepackage{amsthm}
\usepackage{amsfonts}
\usepackage{changepage}
\usepackage{lineno}
\usepackage[plain]{algorithm}
\usepackage{algpseudocode}
\usepackage{hyperref}
\usepackage{tikz}

\topmargin=-0.45in
\evensidemargin=0in
\oddsidemargin=0in
\textwidth=6.5in
\textheight=9.0in
\headsep=0.25in

\linespread{1.1}

\pagestyle{fancy}
\lhead{\hmwkAuthorName}
\chead{\hmwkClass\ (\hmwkClassInstructor\ \hmwkClassTime): \hmwkTitle}
\rhead{\firstxmark}
\lfoot{\lastxmark}
\cfoot{}

\renewcommand\headrulewidth{0.4pt}
\renewcommand\footrulewidth{0.4pt}

\setlength{\floatsep}{100pt}
\renewcommand{\algorithmicrequire}{\textbf{Input:}}
\renewcommand{\algorithmicensure}{\textbf{Output:}}
\algrenewcomment[1]{\hfill // #1}

\setlength\parindent{0pt}

\hypersetup{colorlinks=true}

\newcommand{\enterProblemHeader}[1]{
    \nobreak\extramarks{}{Problem \arabic{#1} continued on next page\ldots}\nobreak{}
    \nobreak\extramarks{Problem \arabic{#1} (continued)}{Problem \arabic{#1} continued on next page\ldots}\nobreak{}
}

\newcommand{\exitProblemHeader}[1]{
    \nobreak\extramarks{Problem \arabic{#1} (continued)}{Problem \arabic{#1} continued on next page\ldots}\nobreak{}
    \stepcounter{#1}
    \nobreak\extramarks{Problem \arabic{#1}}{}\nobreak{}
}

\setcounter{secnumdepth}{0}
\newcounter{partCounter}
\newcounter{homeworkProblemCounter}
\setcounter{homeworkProblemCounter}{1}
\nobreak\extramarks{Problem \arabic{homeworkProblemCounter}}{}\nobreak{}

\newenvironment{homeworkProblem}{
    \section{Problem \arabic{homeworkProblemCounter}}
    \setcounter{partCounter}{1}
    \enterProblemHeader{homeworkProblemCounter}
}{
    \exitProblemHeader{homeworkProblemCounter}
}

\newcommand{\hmwkTitle}{Homework\ \#2}
\newcommand{\hmwkDueDate}{January 29, 2014}
\newcommand{\hmwkClass}{Stat330}
\newcommand{\hmwkClassTime}{Section A}
\newcommand{\hmwkClassInstructor}{Mr. Lanker}
\newcommand{\hmwkAuthorName}{Josh Davis}

\title{
    \vspace{2in}
    \textmd{\textbf{\hmwkClass:\ \hmwkTitle}}\\
    \normalsize\vspace{0.1in}\small{Due\ on\ \hmwkDueDate at 3:10pm}\\
    \vspace{0.1in}\large{\textit{\hmwkClassInstructor\ \hmwkClassTime}}
    \vspace{3in}
}

\author{\textbf{\hmwkAuthorName}}
\date{}

\newcommand{\alg}[1]{\textsc{\bfseries \footnotesize #1}}
\newcommand{\deriv}[1]{\frac{\mathrm{d}}{\mathrm{d}x} (#1)}
\newcommand{\dx}{\mathrm{d}x}
\newcommand{\solution}{\textbf{\large Solution}}

\renewcommand{\part}[1]{\textbf{\large Part \Alph{partCounter}}\stepcounter{partCounter}\\}


\begin{document}

\maketitle

\pagebreak


\begin{homeworkProblem}
    Binomial theorem.
    \\

    \part

    What is the coefficient of \(x^5 y^3\)?
    \\

    \solution

    \[
        c \cdot x^5 y^3 = \binom{8}{3} x^5 y^3= \frac{8!}{5!3!} x^5 y^3 = 56 x^5 y^3
    \]

    \part

    What is the coefficient of \(x^3 y^5\)?
    \\

    \solution

    \[
        c \cdot x^3 y^5 = \binom{8}{5} x^3 y^5= \frac{8!}{5!3!} x^3 y^5 = 56 x^3 y^5
    \]
\end{homeworkProblem}

\pagebreak

\begin{homeworkProblem}
    Simpson's Paradox.
    \\

    \part

    A black urn contains 5 red and 6 green balls and a white urn contains 3 red
    and 4 green balls. You are allowed to choose an urn and then choose a ball
    at random from the urn. Assume that choosing any ball in the urn is equally
    likely. If you choose a red ball you ge a prize. Which urn should you
    choose to draw from?
    \\

    \solution

    For the black urn, the probability of getting a red is:

    \[
        \Pr(\mbox{black})
        = \frac{5 \mbox{ red}}{5 + 6 \mbox{ total}}
        = \frac{5}{11} \approx .455
    \]

    \[
        \Pr(\mbox{white})
        = \frac{3 \mbox{ red}}{3 + 4 \mbox{ total}}
        = \frac{3}{7} \approx .429
    \]

    Therefore we have a higher probability of getting a red from the black urn
    and should choose from that one.
    \\

    \part

    A second black urn contains 6 red and 3 green balls and a second white urn
    contains 9 red and 5 green balls.
    \\

    \solution

    \[
        \Pr(\mbox{2nd black})
        = \frac{6 \mbox{ red}}{6 + 3 \mbox{ total}}
        = \frac{6}{9} \approx .667
    \]

    \[
        \Pr(\mbox{2nd white})
        = \frac{9 \mbox{ red}}{9 + 5 \mbox{ total}}
        = \frac{9}{14} \approx .643
    \]

    The 2nd black has a higher probability than the 2nd white so we should
    choose from that one.
    \\

    \part

    The two black urns are combined as well as the two white urns.
    \\

    \solution

    \[
        \Pr(\mbox{combined black})
        = \frac{5 + 6 \mbox{ red}}{11 + 9 \mbox{ total}}
        = \frac{11}{20} \approx .55
    \]

    \[
        \Pr(\mbox{combined white})
        = \frac{3 + 9 \mbox{ red}}{7 + 14 \mbox{ total}}
        = \frac{12}{21} \approx .571
    \]

    The combined white urn has a higher probability of getting a red so we
    should choose that one.
\end{homeworkProblem}

\pagebreak

\begin{homeworkProblem}
    A group of 4 undergraduates and 5 graduates are available to fill four
    student government posts.
    \\

    \part

    Find the probability that at least three undergraduates will be among the
    four chosen assuming the selection was random?
    \\

    \solution
    \\

    \part

    What is the probability of no undergraduates being selected again assuming
    the selection was random?
    \\

    \solution
    \\

    \part

    Later we will talk about a \(p\)-value, which represents the probability
    happening under certain model assumptions. Here our model assumptions are
    that the selection process was purely random. It is common that if the
    \(p\)-value is very low, say less than .05, we reject that our model
    assumptions are true. Our \(p\)-value here is the result from part b. What
    does this value say about the model statement that the selections were made
    at random?
    \\

    \solution

\end{homeworkProblem}

\pagebreak

\begin{homeworkProblem}
    Suppose that there are two events \(A\) and \(B\) in \(\Omega\) such that
    \(\Pr(A) > 0\) and \(\Pr(B) > 0\).  Further, suppose that the two events are
    mutually exclusive. Can they also be independent? Explain.
    \\

    \solution

\end{homeworkProblem}

\pagebreak

\begin{homeworkProblem}
    Total 60 students attending University. 9 were living off campus, 36 were
    undergrads, 3 were undergrads living off campus.
    \\

    Let \(A\) be the event denoting undergraduates and \(B\) denote living off campus.
    \[
        \begin{split}
            A &= 36
            \\
            B &= 9
            \\
            \overline{A} &= 60 - 36 = 24
            \\
            \overline{B} &= 60 - 9 = 51
            \\
            A \cap B &= 3
        \end{split}
    \]

    \part

    Number of students who were undergrads living on campus.
    \\

    \solution

    \[
        \begin{split}
            A \cap \overline{B} &= A \setminus B
            \\
            &= 36 - 9 + 3
            \\
            &= 30
        \end{split}
    \]
    \\

    \part

    Number of students who were graduate students living on campus.
    \\

    \solution

    \[
        \begin{split}
            \overline{A} \cap \overline{B} &= \overline{A} \setminus B
            \\
            &= 24 - 9 + 3
            \\
            &= 18
        \end{split}
    \]
\end{homeworkProblem}

\pagebreak

\begin{homeworkProblem}
    Let \(A\) be the event that processor 1 is in use and \(B\) be the event
    that processor 2 is in use.
    \\

    \part

    Calculate \(\Pr(A \mid B)\).
    \\

    \solution
    \\

    \part

    What is \(\Pr(A)\)?
    \\

    \solution
    \\

    \part

    Are \(A\) and \(B\) independent?
    \\

    \solution
    \\

    \part

    Calculate \(\Pr(B \mid A)\).
    \\

    \solution
    \\

    \part

    Show that \(\Pr(A \mid B) \cdot \Pr(B) = \Pr(B \mid A) \cdot \Pr(A)\).
    \\

    \solution
    \\
\end{homeworkProblem}

\end{document}
