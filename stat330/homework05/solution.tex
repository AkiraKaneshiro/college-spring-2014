\documentclass{article}

\usepackage{fancyhdr}
\usepackage{extramarks}
\usepackage{amsmath}
\usepackage{amsthm}
\usepackage{amsfonts}
\usepackage{multirow}

\topmargin=-0.45in
\evensidemargin=0in
\oddsidemargin=0in
\textwidth=6.5in
\textheight=9.0in
\headsep=0.25in

\linespread{1.1}

\pagestyle{fancy}
\lhead{\hmwkAuthorName}
\chead{\hmwkClass\ (\hmwkClassInstructor\ \hmwkClassTime): \hmwkTitle}
\rhead{\firstxmark}
\lfoot{\lastxmark}
\cfoot{\thepage}

\renewcommand\headrulewidth{0.4pt}
\renewcommand\footrulewidth{0.4pt}

\setlength\parindent{0pt}

\newcommand{\enterProblemHeader}[1]{
    \nobreak\extramarks{}{Problem \arabic{#1} continued on next page\ldots}\nobreak{}
    \nobreak\extramarks{Problem \arabic{#1} (continued)}{Problem \arabic{#1} continued on next page\ldots}\nobreak{}
}

\newcommand{\exitProblemHeader}[1]{
    \nobreak\extramarks{Problem \arabic{#1} (continued)}{Problem \arabic{#1} continued on next page\ldots}\nobreak{}
    \stepcounter{#1}
    \nobreak\extramarks{Problem \arabic{#1}}{}\nobreak{}
}

\setcounter{secnumdepth}{0}
\newcounter{partCounter}
\newcounter{homeworkProblemCounter}
\setcounter{homeworkProblemCounter}{1}
\nobreak\extramarks{Problem \arabic{homeworkProblemCounter}}{}\nobreak{}

\newenvironment{homeworkProblem}{
    \section{Problem \arabic{homeworkProblemCounter}}
    \setcounter{partCounter}{1}
    \enterProblemHeader{homeworkProblemCounter}
}{
    \exitProblemHeader{homeworkProblemCounter}
}

\newcommand{\hmwkTitle}{Homework\ \#5}
\newcommand{\hmwkDueDate}{February 19, 2014}
\newcommand{\hmwkClass}{Stat330}
\newcommand{\hmwkClassTime}{Section A}
\newcommand{\hmwkClassInstructor}{Mr. Lanker}
\newcommand{\hmwkAuthorName}{Josh Davis}

\title{
    \vspace{2in}
    \textmd{\textbf{\hmwkClass:\ \hmwkTitle}}\\
    \normalsize\vspace{0.1in}\small{Due\ on\ \hmwkDueDate\ at 3:10pm}\\
    \vspace{0.1in}\large{\textit{\hmwkClassInstructor\ \hmwkClassTime}}
    \vspace{3in}
}

\author{\textbf{\hmwkAuthorName}}
\date{}

\newcommand{\alg}[1]{\textsc{\bfseries \footnotesize #1}}
\newcommand{\deriv}[1]{\frac{\mathrm{d}}{\mathrm{d}x} (#1)}
\newcommand{\dx}{\mathrm{d}x}
\newcommand{\dt}{\mathrm{d}t}
\newcommand{\solution}{\textbf{\large Solution}}

\renewcommand{\part}[1]{\textbf{\large Part \Alph{partCounter}}\stepcounter{partCounter}\\}

\newcommand{\E}{\mathrm{E}}
\newcommand{\Var}{\mathrm{Var}}
\newcommand{\Cov}{\mathrm{Cov}}
\newcommand{\dist}[1]{\sim \mathrm{#1}}

\begin{document}

\maketitle

\pagebreak

\begin{homeworkProblem}
    Problem 3.15 from Baron.
    \\

    Let \(X\) and \(Y\) be the number of hardware failures in two computer labs
    in a given month.  The joint distribution of \(X\) and \(Y\) is givin in
    the table below:

    \begin{table}[ht]
        \centering
        \begin{tabular}{|c | c || c | c | c | c |}
            \hline
            \multicolumn{2}{|c||}{} & \multicolumn{3}{c|}{$x$} &
            \\
            \cline{3-6}
            \multicolumn{2}{|c||}{$P(x, y)$} & 0 & 1 & 2 & $P_Y(y)$
            \\
            \hline
            \hline
            \multirow{3}{*}{$y$} & 0 & 0.52 & 0.20 & 0.04 & 0.76
            \\
            \cline{2-6}
            & 1 & 0.14 & 0.02 & 0.01 & 0.17
            \\
            \cline{2-6}
            & 2 & 0.06 & 0.01 & 0.00 & 0.07
            \\
            \hline
            \multicolumn{2}{|c||}{$P_X(x)$} & 0.72 & 0.23 & 0.05 & 1.00
            \\
            \hline
        \end{tabular}
    \end{table}

    \part

    Compute the probability of at least one hardware failure.
    \\

    \solution

    This means that either one component in either lab fails. Thus it would be
    equal to the complement of the probability that \textbf{no} components fail.
    \\

    This gives us:

    \[
        P_X(X \geq 1) + P_Y(Y \geq 1) = 1 - P(X = 0, Y = 0) = 1 - 0.52 = 0.48
    \]

    \part

    From the given distribution, are \(X\) and \(Y\) independent? Why or why
    not?
    \\

    \solution

    To determine if joint probabilities are independent, we just need to see
    if the probability of \(P_{XY}(x, y)\) is equal to the separate probabilties
    for both \(P_X(x)\) and \(P_Y(y)\).
    \\

    Thus using our table, we just have to check if the following equation holds
    \(P(x, y) = P_X(x) P_Y(y)\) for each part in the table. Just by looking at
    the first cell, \(P(0, 0) = 0.52 \neq 0.547 = P_X(0) P_Y(0)\).
    \\

    Thus our variables, \(X\) and \(Y\), aren't independent, they are
    dependent.
\end{homeworkProblem}

\pagebreak

\begin{homeworkProblem}
    Two random variables \(X\) and \(Y\) have the joint distribution defined by
    the table:

    \begin{table}[ht]
        \centering
        \begin{tabular}{|c | c || c | c | c | c |}
            \hline
            \multicolumn{2}{|c||}{} & \multicolumn{3}{c|}{$x$} &
            \\
            \cline{3-6}
            \multicolumn{2}{|c||}{$P(x, y)$} & 0 & 1 & 2 & $P_Y(y)$
            \\
            \hline
            \hline
            \multirow{3}{*}{$y$} & 0 & 0.4 & 0.2 & 0.1 & 0.7
            \\
            \cline{2-6}
            & 1 & 0 & 0.2 & 0 & 0.2
            \\
            \cline{2-6}
            & 2 & 0 & 0 & 0.1 & 0.1
            \\
            \hline
            \multicolumn{2}{|c||}{$P_X(x)$} & 0.4 & 0.4 & 0.2 & 1.00
            \\
            \hline
        \end{tabular}
    \end{table}

    \part

    Find the probability mass function of a new random variable \(U = 2X -
    Y\) by looking at the possible values for \(u\) based on the defined
    joint distribution. Hint: the value of \(u = 0\) has the probability
    0.4.
    \\

    \solution

    Using the equation and the above table, this gives us the following:

    \begin{table}[ht]
        \centering
        \begin{tabular}{|c | c || c | c | c |}
            \hline
            \multicolumn{2}{|c||}{} & \multicolumn{3}{c|}{$x$}
            \\
            \cline{3-5}
            \multicolumn{2}{|c||}{$U$} & 0 & 1 & 2
            \\
            \hline
            \hline
            \multirow{3}{*}{$y$} & 0 & 0 & 2 & 4
            \\
            \cline{2-5}
            & 1 & -1 & 1 & 3
            \\
            \cline{2-5}
            & 2 & -2 & 0 & 2
            \\
            \hline
        \end{tabular}
    \end{table}

    \begin{table}[ht]
        \centering
        \begin{tabular}{c || c | c | c | c | c | c | c}
            \(u\) & -2 & -1 & 0 & 1 & 2 & 3 & 4
            \\
            \hline
            \hline
            Values
            & \(P(0, 2)\)
            & \(P(0, 1)\)
            & \(P(0, 0) + P(1, 2)\)
            & \(P(1, 1)\)
            & \(P(1, 0) + P(2, 2)\)
            & \(P(2, 1)\)
            & \(P(2, 0)\)
            \\
            \hline
            Sum
            & 0
            & 0
            & 0.4 + 0
            & 0.2
            & 0.2 + 0.1
            & 0
            & 0.1
            \\
            \hline
            \hline
            \(P(u)\)
            & 0
            & 0
            & 0.4
            & 0.2
            & 0.3
            & 0
            & 0.1
        \end{tabular}
    \end{table}

    \part

    Find \(\E[U]\) using \(P(U)\) probability mass function.
    \\

    \solution

    Since it is a discrete random variable, our expected value is:

    \[
        \begin{split}
            \E[U] &= \sum_{u = -2}^{4} u P(u)
            \\
            &= -2(0) + -1(0) + 0(0.4) + 1(0.2) + 2(0.3) + 3(0) + 4(0.1)
            \\
            &= 1.2
        \end{split}
    \]

    \pagebreak

    \part

    The expected values for \(X\) and \(Y\) are \(\E[X] = 0.8\)
    and \(\E[Y] = 0.4\). This time calculate \(\E[U]\) using properties of
    expected values (see pg. 49).
    \\

    \solution

    Using the properties on page 49, we get:

    \[
        \begin{split}
            \E[U] &= \E[2X - Y]
            \\
            &= \E[2X] - \E[Y]
            \\
            &= 2\E[X] - \E[Y]
            \\
            &= 2(.8) - (.4)
            \\
            &= 1.2
        \end{split}
    \]

    which agrees with part (b).
    \\

    \part

    Calculate \(P(X = 2 \mid U \geq 2)\).
    \\

    \solution

    We want to find the probability that given \(U \geq 2\), that \(X = 2\).
    Using our table, this gives us:

    \[
        \begin{split}
            P(X = 2 \mid U \geq 2)
            &= \frac{
                P(X = 2 \cap U \geq 2)
            }{
                P(U \geq 2)
            }
            \\
            &= \frac{
                P(2, 2) + P(2, 1) + P(2, 0)
            }{
                P(U = 2) + P(U = 3) + P(U = 4)
            }
            \\
            &= \frac{
                0.1 + 0.1 + 0.1
            }{
                0.3 + 0 + 0.1
            }
            \\
            &= 0.75
        \end{split}
    \]
\end{homeworkProblem}

\pagebreak

\begin{homeworkProblem}
    The time in minutes for a certain system to reboot can be modeled with a
    continuous random variable \(T\) that has the following pdf:

    \[
        f(t) = \left\{
            \begin{split}
                    {C(2 - t)}^2 \quad & 0 \leq t \leq 2
                    \\
                    0 \quad & \mbox{any other } t
            \end{split}
        \right.
    \]

    \part

    Compute \(C\).
    \\

    \solution

    Since this is a probability density, we know that the total area of any
    probability density, is equal to one. Thus we get:

    \[
        \begin{split}
            F(t) &= \int_0^2 C \cdot {(2 - t)}^2 \dt
            \\
            &= C \left[ \int_0^2 t^2 - 4t + 4\ \dt \right]
            \\
            &= C \left[ \frac{1}{3}t^3 - 2t^2 + 4t + c \right|_0^2
            \\
            &= C \left[ \frac{1}{3}(2)^3 - 2(2)^2 + 4(2) + c \right]
            - \left[ \frac{1}{3}(0)^3 - 2(0)^2 + 4(0) + c \right]
            \\
            &= C \left[
                (\frac{8}{3} - 8 + 8 + c) - (c)
            \right]
            \\
            &= C \frac{8}{3}
        \end{split}
    \]

    Since this is equal to 1, we can see that \(C = 3/8\).
    \\

    \part

    Compute \(\E[T]\).
    \\

    \solution

    Using \(C = 3/8\) from the previous problem, we get:

    \[
        \begin{split}
            \E[f(t)] &= \frac{3}{8} \int_0^2 t \cdot {(2 - t)}^2 \dt
            \\
            &= \frac{3}{8} \left[ \int_0^2 t^3 - 4t^2 + 4t\ \dt \right]
            \\
            &= \frac{3}{8} \left[ \frac{1}{4}t^4 - \frac{4}{3}t^3 + 2t^2\ \right|_0^2
            \\
            &= \frac{3}{8} \left[ \frac{1}{4}t^4 - \frac{4}{3}t^3 + 2t^2\ \right|_0^2
            \\
            &= \frac{3}{8} \left[ \frac{1}{4}(2)^4 - \frac{4}{3}(2)^3 + 2(2)^2\ \right|_0^2
            \\
            &= \frac{3}{8} \left[ 4 - \frac{32}{3} + 8\ \right] = \frac{1}{2}
        \end{split}
    \]

    \part

    Compute \(\Var[T]\).
    \\

    \solution

    We know that \(\Var[T] = \E[(T - \E[T])^2]\) or \(\Var[T] = \E[T^2] -
    (\E[T])^2\) which gives us:

    \[
        \begin{split}
            \Var[f(t)] &= \frac{3}{8} \int_0^2 t^2 \cdot {(2 - t)}^2 \dt - \left(\frac{1}{2}\right)^2
            \\
            &= \frac{3}{8} \left[ \int_0^2 t^4 - 4t^3 + 4t^2\ \dt \right] - \frac{1}{4}
            \\
            &= \frac{3}{8} \left[ \frac{1}{5}t^5 - t^4 + \frac{4}{3}t^3\ \right|_0^2 - \frac{1}{4}
            \\
            &= \frac{3}{8} \left[ \frac{1}{5}(2)^5 - (2)^4 + \frac{4}{3}(2)^3\ \right| - \frac{1}{4}
            \\
            &= \frac{3}{8} \left[ \frac{32}{5} - 16 + \frac{32}{3} \right] - \frac{1}{4}
            \\
            &= \frac{3}{8} \left[ \frac{16}{15} \right] - \frac{1}{4}
            \\
            &= \frac{3}{20}
            \\
            &= .15
        \end{split}
    \]

    \part

    Compute the probability that it takes between 1 and 2 minutes to reboot.
    \\

    \solution

    Since probability is area of our pdf, we just need to calculate the
    integral from 1 to 2. Thus:

    \[
        \begin{split}
            F(t) &= \frac{3}{8} \int_1^2 {(2 - t)}^2 \dt
            \\
            &= \frac{3}{8} \left[ \int_1^2 t^2 - 4t + 4\ \dt \right]
            \\
            &= \frac{3}{8} \left[ \frac{1}{3}t^3 - 2t^2 + 4t + c \right|_1^2
            \\
            &= \frac{3}{8} \left[ \left(\frac{1}{3}(2)^3 - 2(2)^2 + 4(2) + c \right)
            - \left( \frac{1}{3}(1)^3 - 2(1)^2 + 4(1) + c \right) \right]
            \\
            &= \frac{3}{8} \left[ \left(\frac{8}{3} - 8 + 8 + c \right)
            - \left( \frac{1}{3} - 2 + 4 + c \right) \right]
            \\
            &= \frac{3}{8} \left[ \left(\frac{8}{3}\right)
            - \left( \frac{7}{3}\right) \right]
            \\
            &= \frac{3}{8} \left[ \frac{1}{3} \right]
            \\
            &= \frac{1}{8}
            \\
            &= 0.125
        \end{split}
    \]
\end{homeworkProblem}

\pagebreak

\begin{homeworkProblem}

\end{homeworkProblem}

\begin{homeworkProblem}

\end{homeworkProblem}

\pagebreak

\begin{homeworkProblem}

\end{homeworkProblem}

\pagebreak

\begin{homeworkProblem}

\end{homeworkProblem}

\pagebreak

\begin{homeworkProblem}
    If \(X\) is a continuous random variable that follows a normal probability
    distribution with \(\mu = 100\) points and \(\sigma = 15\), then find the
    probability that \(X\) is:
    \\

    \part

    Less than 106.
    \\

    \solution

    Solution.
    \\

    \part

    Greater than 88.
    \\

    \solution

    Solution.
    \\

    \part

    Between 85 and 115.
    \\

    \solution

    Solution.
    \\

    \part

    Find the point value that separates the lower half and upper half of values
    of \(X\), or the median value \(c\) where \(P(X < c) = 0.50\).
    \\

    \solution

    Solution.
    \\

    \part

    Find the point value that separates the lower quarter from the upper three
    quarters of values of \(X\), or where \(P(X < c) = 0.25\).
    \\

    \solution

    Solution.
    \\

    \part

    Find the point value that separates the lower three quarters from the upper
    quarter of values of \(X\), or where \(P(X < c) = 0.75\).
    \\

    \solution

    Solution.
\end{homeworkProblem}

\pagebreak

\begin{homeworkProblem}
    Problem 4.22 from Baron.
    \\

    Refer to the country in example 4.11 on pg. 91, where household incomes
    follow Normal distribution with \(\mu = 900\) coins and \(\sigma = 200\)
    coins.
    \\

    \part

    A recent economic reform made households with the income below 640 coins
    qualify for a free bottle of milk at every breakfast. What portion of the
    population qualifies for a free bottle of milk?
    \\

    \solution

    Solution.
    \\

    \part

    Moreover, households with an income within the lowest 5\% of the polulation
    are entitled to a free sandwich. What income qualifies a household to
    receive free sandwiches?
    \\

    \solution

    Solution.
\end{homeworkProblem}

\pagebreak

\begin{homeworkProblem}
    The average height of professinoal basketball players is around 6 feet and
    7 inches, and the standard deviation is 3.89 inches. Assuming Normal
    distribution of heights within this group:
    \\

    \part

    What percent of professional backetball players are taller than 7 feet?
    \\

    \solution

    Solution.
    \\

    \part

    If your favorite player is within the tallest 20\% of all players, what can
    his height be?
    \\

    \solution

    Solution.
\end{homeworkProblem}

\pagebreak

\end{document}
