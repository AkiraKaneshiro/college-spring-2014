\documentclass{article}

\usepackage{fancyhdr}
\usepackage{extramarks}
\usepackage{amsmath}
\usepackage{amsthm}
\usepackage{amsfonts}
\usepackage{multirow}

\topmargin=-0.45in
\evensidemargin=0in
\oddsidemargin=0in
\textwidth=6.5in
\textheight=9.0in
\headsep=0.25in

\linespread{1.1}

\pagestyle{fancy}
\lhead{\hmwkAuthorName}
\chead{\hmwkClass\ (\hmwkClassInstructor\ \hmwkClassTime): \hmwkTitle}
\rhead{\firstxmark}
\lfoot{\lastxmark}
\cfoot{\thepage}

\renewcommand\headrulewidth{0.4pt}
\renewcommand\footrulewidth{0.4pt}

\setlength\parindent{0pt}

\newcommand{\enterProblemHeader}[1]{
    \nobreak\extramarks{}{Problem \arabic{#1} continued on next page\ldots}\nobreak{}
    \nobreak\extramarks{Problem \arabic{#1} (continued)}{Problem \arabic{#1} continued on next page\ldots}\nobreak{}
}

\newcommand{\exitProblemHeader}[1]{
    \nobreak\extramarks{Problem \arabic{#1} (continued)}{Problem \arabic{#1} continued on next page\ldots}\nobreak{}
    \stepcounter{#1}
    \nobreak\extramarks{Problem \arabic{#1}}{}\nobreak{}
}

\setcounter{secnumdepth}{0}
\newcounter{partCounter}
\newcounter{homeworkProblemCounter}
\setcounter{homeworkProblemCounter}{1}
\nobreak\extramarks{Problem \arabic{homeworkProblemCounter}}{}\nobreak{}

\newenvironment{homeworkProblem}{
    \section{Problem \arabic{homeworkProblemCounter}}
    \setcounter{partCounter}{1}
    \enterProblemHeader{homeworkProblemCounter}
}{
    \exitProblemHeader{homeworkProblemCounter}
}

\newcommand{\hmwkTitle}{Homework\ \#5}
\newcommand{\hmwkDueDate}{February 19, 2014}
\newcommand{\hmwkClass}{Stat330}
\newcommand{\hmwkClassTime}{Section A}
\newcommand{\hmwkClassInstructor}{Mr. Lanker}
\newcommand{\hmwkAuthorName}{Josh Davis}

\title{
    \vspace{2in}
    \textmd{\textbf{\hmwkClass:\ \hmwkTitle}}\\
    \normalsize\vspace{0.1in}\small{Due\ on\ \hmwkDueDate\ at 3:10pm}\\
    \vspace{0.1in}\large{\textit{\hmwkClassInstructor\ \hmwkClassTime}}
    \vspace{3in}
}

\author{\textbf{\hmwkAuthorName}}
\date{}

\newcommand{\alg}[1]{\textsc{\bfseries \footnotesize #1}}
\newcommand{\deriv}[1]{\frac{\mathrm{d}}{\mathrm{d}x} (#1)}
\newcommand{\dx}{\mathrm{d}x}
\newcommand{\dt}{\mathrm{d}t}
\newcommand{\solution}{\textbf{\large Solution}}

\renewcommand{\part}[1]{\textbf{\large Part \Alph{partCounter}}\stepcounter{partCounter}\\}

\newcommand{\E}{\mathrm{E}}
\newcommand{\Var}{\mathrm{Var}}
\newcommand{\Cov}{\mathrm{Cov}}
\newcommand{\dist}[1]{\sim \mathrm{#1}}

\begin{document}

\maketitle

\pagebreak

\begin{homeworkProblem}
    Problem 3.15 from Baron.
    \\

    Let \(X\) and \(Y\) be the number of hardware failures in two computer labs
    in a given month.  The joint distribution of \(X\) and \(Y\) is givin in
    the table below:

    \begin{table}[ht]
        \centering
        \begin{tabular}{|c | c || c | c | c | c |}
            \hline
            \multicolumn{2}{|c||}{} & \multicolumn{3}{c|}{$x$} &
            \\
            \cline{3-5}
            \multicolumn{2}{|c||}{$P(x, y)$} & 0 & 1 & 2 & $P_X(x)$
            \\
            \hline
            \hline
            \multirow{3}{*}{$y$} & 0 & 0.52 & 0.20 & 0.04 & 0.76
            \\
            \cline{2-5}
            & 1 & 0.14 & 0.02 & 0.01 & 0.17
            \\
            \cline{2-5}
            & 2 & 0.06 & 0.01 & 0.00 & 0.07
            \\
            \hline
            \multicolumn{2}{|c||}{$P_Y(y)$} & 0.72 & 0.23 & 0.05 & 1.00
            \\
            \hline
        \end{tabular}
    \end{table}

    \part

    Compute the probability of at least one hardware failure.
    \\

    \solution

    This means that either one component in either lab fails. Thus it would be
    equal to the complement of the probability that \textbf{no} components fail.
    \\

    This gives us:

    \[
        P_X(X \geq 1) + P_Y(Y \geq 1) = 1 - P(X = 0, Y = 0) = 1 - 0.52 = .48
    \]

    \part

    From the given distribution, are \(X\) and \(Y\) independent? Why or why not?
    \\

    \solution

    To determine if joint probabilities are independent, we just need to see
    if the probability of \(P_{XY}(x, y)\) is equal to the separate probabilties
    for both \(P_X(x)\) and \(P_Y(y)\).
    \\

    Thus using our table, we have:
    \[
        P(x, y) = P_X(x) P_Y(y)
    \]
\end{homeworkProblem}

\pagebreak

\begin{homeworkProblem}

\end{homeworkProblem}

\pagebreak

\begin{homeworkProblem}
    The time in minutes for a certain system to reboot can be modeled with a
    continuous random variable \(T\) that has the following pdf:

    \[
        f(t) = \left\{
            \begin{split}
                    {C(2 - t)}^2 \quad & 0 \leq t \leq 2
                    \\
                    0 \quad & \mbox{any other } t
            \end{split}
        \right.
    \]

    \part

    Compute \(C\).
    \\

    \solution

    Since this is a probability density, we know that the total area of any
    probability density, is equal to one. Thus we get:

    \[
        \begin{split}
            F(t) &= \int_0^2 C \cdot {(2 - t)}^2 \dt
            \\
            &= C \left[ \int_0^2 t^2 - 4t + 4\ \dt \right]
            \\
            &= C \left[ \frac{1}{3}t^3 - 2t^2 + 4t + c \right|_0^2
            \\
            &= C \left[ \frac{1}{3}(2)^3 - 2(2)^2 + 4(2) + c \right]
            - \left[ \frac{1}{3}(0)^3 - 2(0)^2 + 4(0) + c \right]
            \\
            &= C \left[
                (\frac{8}{3} - 8 + 8 + c) - (c)
            \right]
            \\
            &= C \frac{8}{3}
        \end{split}
    \]

    Since this is equal to 1, we can see that \(C = 3/8\).
    \\

    \part

    Compute \(\E[T]\).
    \\

    \solution

    Using \(C = 3/8\) from the previous problem, we get:

    \[
        \begin{split}
            \E[f(t)] &= \frac{3}{8} \int_0^2 t \cdot {(2 - t)}^2 \dt
            \\
            &= \frac{3}{8} \left[ \int_0^2 t^3 - 4t^2 + 4t\ \dt \right]
            \\
            &= \frac{3}{8} \left[ \frac{1}{4}t^4 - \frac{4}{3}t^3 + 2t^2\ \right|_0^2
            \\
            &= \frac{3}{8} \left[ \frac{1}{4}t^4 - \frac{4}{3}t^3 + 2t^2\ \right|_0^2
            \\
            &= \frac{3}{8} \left[ \frac{1}{4}(2)^4 - \frac{4}{3}(2)^3 + 2(2)^2\ \right|_0^2
            \\
            &= \frac{3}{8} \left[ 4 - \frac{32}{3} + 8\ \right] = \frac{1}{2}
        \end{split}
    \]

    \part

    Compute \(\Var[T]\).
    \\

    \solution

    We know that \(\Var[T] = \E[(T - \E[T])^2]\) or \(\Var[T] = \E[T^2] -
    (\E[T])^2\) which gives us:

    \[
        \begin{split}
            \Var[f(t)] &= \frac{3}{8} \int_0^2 t^2 \cdot {(2 - t)}^2 \dt - \left(\frac{1}{2}\right)^2
            \\
            &= \frac{3}{8} \left[ \int_0^2 t^4 - 4t^3 + 4t^2\ \dt \right] - \frac{1}{4}
            \\
            &= \frac{3}{8} \left[ \frac{1}{5}t^5 - t^4 + \frac{4}{3}t^3\ \right|_0^2 - \frac{1}{4}
            \\
            &= \frac{3}{8} \left[ \frac{1}{5}(2)^5 - (2)^4 + \frac{4}{3}(2)^3\ \right| - \frac{1}{4}
            \\
            &= \frac{3}{8} \left[ \frac{32}{5} - 16 + \frac{32}{3} \right] - \frac{1}{4}
            \\
            &= \frac{3}{8} \left[ \frac{16}{15} \right] - \frac{1}{4}
            \\
            &= \frac{3}{20}
            \\
            &= .15
        \end{split}
    \]

    \part

    Compute the probability that it takes between 1 and 2 minutes to reboot.
    \\

    \solution

    Since probability is area of our pdf, we just need to calculate the
    integral from 1 to 2. Thus:

    \[
        \begin{split}
            F(t) &= \frac{3}{8} \int_1^2 {(2 - t)}^2 \dt
            \\
            &= \frac{3}{8} \left[ \int_1^2 t^2 - 4t + 4\ \dt \right]
            \\
            &= \frac{3}{8} \left[ \frac{1}{3}t^3 - 2t^2 + 4t + c \right|_1^2
            \\
            &= \frac{3}{8} \left[ \left(\frac{1}{3}(2)^3 - 2(2)^2 + 4(2) + c \right)
            - \left( \frac{1}{3}(1)^3 - 2(1)^2 + 4(1) + c \right) \right]
            \\
            &= \frac{3}{8} \left[ \left(\frac{8}{3} - 8 + 8 + c \right)
            - \left( \frac{1}{3} - 2 + 4 + c \right) \right]
            \\
            &= \frac{3}{8} \left[ \left(\frac{8}{3}\right)
            - \left( \frac{7}{3}\right) \right]
            \\
            &= \frac{3}{8} \left[ \frac{1}{3} \right]
            \\
            &= \frac{1}{8}
            \\
            &= 0.125
        \end{split}
    \]
\end{homeworkProblem}

\pagebreak

\begin{homeworkProblem}

\end{homeworkProblem}

\begin{homeworkProblem}

\end{homeworkProblem}

\pagebreak

\begin{homeworkProblem}

\end{homeworkProblem}

\pagebreak

\begin{homeworkProblem}

\end{homeworkProblem}

\pagebreak

\begin{homeworkProblem}

\end{homeworkProblem}

\pagebreak

\begin{homeworkProblem}

\end{homeworkProblem}

\pagebreak

\begin{homeworkProblem}

\end{homeworkProblem}

\pagebreak

\end{document}
