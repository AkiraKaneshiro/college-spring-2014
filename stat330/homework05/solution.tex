\documentclass{article}

\usepackage{fancyhdr}
\usepackage{extramarks}
\usepackage{amsmath}
\usepackage{amsthm}
\usepackage{amsfonts}
\usepackage{tikz}
\usepackage{pgfplots}

\topmargin=-0.45in
\evensidemargin=0in
\oddsidemargin=0in
\textwidth=6.5in
\textheight=9.0in
\headsep=0.25in

\linespread{1.1}

\pagestyle{fancy}
\lhead{\hmwkAuthorName}
\chead{\hmwkClass\ (\hmwkClassInstructor\ \hmwkClassTime): \hmwkTitle}
\rhead{\firstxmark}
\lfoot{\lastxmark}
\cfoot{\thepage}

\renewcommand\headrulewidth{0.4pt}
\renewcommand\footrulewidth{0.4pt}

\setlength\parindent{0pt}

\newcommand{\enterProblemHeader}[1]{
    \nobreak\extramarks{}{Problem \arabic{#1} continued on next page\ldots}\nobreak{}
    \nobreak\extramarks{Problem \arabic{#1} (continued)}{Problem \arabic{#1} continued on next page\ldots}\nobreak{}
}

\newcommand{\exitProblemHeader}[1]{
    \nobreak\extramarks{Problem \arabic{#1} (continued)}{Problem \arabic{#1} continued on next page\ldots}\nobreak{}
    \stepcounter{#1}
    \nobreak\extramarks{Problem \arabic{#1}}{}\nobreak{}
}

\setcounter{secnumdepth}{0}
\newcounter{partCounter}
\newcounter{homeworkProblemCounter}
\setcounter{homeworkProblemCounter}{1}
\nobreak\extramarks{Problem \arabic{homeworkProblemCounter}}{}\nobreak{}

\newenvironment{homeworkProblem}{
    \section{Problem \arabic{homeworkProblemCounter}}
    \setcounter{partCounter}{1}
    \enterProblemHeader{homeworkProblemCounter}
}{
    \exitProblemHeader{homeworkProblemCounter}
}

\newcommand{\hmwkTitle}{Homework\ \#5}
\newcommand{\hmwkDueDate}{February 19, 2014}
\newcommand{\hmwkClass}{Stat330}
\newcommand{\hmwkClassTime}{Section A}
\newcommand{\hmwkClassInstructor}{Mr. Lanker}
\newcommand{\hmwkAuthorName}{Josh Davis}

\title{
    \vspace{2in}
    \textmd{\textbf{\hmwkClass:\ \hmwkTitle}}\\
    \normalsize\vspace{0.1in}\small{Due\ on\ \hmwkDueDate\ at 3:10pm}\\
    \vspace{0.1in}\large{\textit{\hmwkClassInstructor\ \hmwkClassTime}}
    \vspace{3in}
}

\author{\textbf{\hmwkAuthorName}}
\date{}

\newcommand{\alg}[1]{\textsc{\bfseries \footnotesize #1}}
\newcommand{\deriv}[1]{\frac{\mathrm{d}}{\mathrm{d}x} (#1)}
\newcommand{\dx}{\mathrm{d}x}
\newcommand{\solution}{\textbf{\large Solution}}

\renewcommand{\part}[1]{\textbf{\large Part \Alph{partCounter}}\stepcounter{partCounter}\\}

\newcommand{\E}{\mathrm{E}}
\newcommand{\Var}{\mathrm{Var}}
\newcommand{\Cov}{\mathrm{Cov}}
\newcommand{\dist}[1]{\sim \mathrm{#1}}

\begin{document}

\maketitle

\pagebreak

\begin{homeworkProblem}

\end{homeworkProblem}

\pagebreak

\begin{homeworkProblem}

\end{homeworkProblem}

\pagebreak

\begin{homeworkProblem}
    The time in minutes for a certain system to reboot can be modeled
    with a continous random variable \(T\) that has the following pdf:

    \[
        f(t) = \left\{
            \begin{split}
                    {C(2 - t)}^2 \quad & 0 \leq t \leq 2
                    \\
                    0 \quad & \mbox{any other } t
            \end{split}
        \right.
    \]

    \part

    Compute \(C\).
    \\

    \solution

    Since this is a probability density, we know that the total area of any
    probability density, is equal to one. Thus we get:

    \[
        \begin{split}
            F(x) &= \int_0^2 C \cdot {(2 - t)}^2 \dx
            \\
            &= C \left[ \int_0^2 t^2 - 4t + 4\ \dx \right]
            \\
            &= C \left[ \frac{1}{3}t^3 - 2t^2 + 4t + c \right|_0^2
            \\
            &= C \left[ \frac{1}{3}(2)^3 - 2(2)^2 + 4(2) + c \right]
            - \left[ \frac{1}{3}(0)^3 - 2(0)^2 + 4(0) + c \right]
            \\
            &= C \left[
                (\frac{8}{3} - 8 + 8 + c) - (c)
            \right]
            \\
            &= C \frac{8}{3}
        \end{split}
    \]

    Since this is equal to 1, we can see that \(C = 3/8\).
    \\

    \part

    Compute \(\E[T]\).
    \\

    \solution

    Solution.
    \\

    \part

    Compute \(\Var[T]\).
    \\

    \solution

    Solution.
    \\

    \part

    Compute the probability that it takes between 1 and 2 minutes to reboot.
    \\

    \solution

    Solution.
    \\
\end{homeworkProblem}

\pagebreak

\begin{homeworkProblem}

\end{homeworkProblem}

\begin{homeworkProblem}

\end{homeworkProblem}

\pagebreak

\begin{homeworkProblem}

\end{homeworkProblem}

\pagebreak

\begin{homeworkProblem}

\end{homeworkProblem}

\pagebreak

\begin{homeworkProblem}

\end{homeworkProblem}

\pagebreak

\begin{homeworkProblem}

\end{homeworkProblem}

\pagebreak

\begin{homeworkProblem}

\end{homeworkProblem}

\pagebreak

\end{document}
