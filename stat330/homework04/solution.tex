\documentclass{article}

\usepackage{fancyhdr}
\usepackage{extramarks}
\usepackage{amsmath}
\usepackage{amsthm}
\usepackage{amsfonts}

\topmargin=-0.45in
\evensidemargin=0in
\oddsidemargin=0in
\textwidth=6.5in
\textheight=9.0in
\headsep=0.25in

\linespread{1.1}

\pagestyle{fancy}
\lhead{\hmwkAuthorName}
\chead{\hmwkClass\ (\hmwkClassInstructor\ \hmwkClassTime): \hmwkTitle}
\rhead{\firstxmark}
\lfoot{\lastxmark}
\cfoot{\thepage}

\renewcommand\headrulewidth{0.4pt}
\renewcommand\footrulewidth{0.4pt}

\setlength\parindent{0pt}

\newcommand{\enterProblemHeader}[1]{
    \nobreak\extramarks{}{Problem \arabic{#1} continued on next page\ldots}\nobreak{}
    \nobreak\extramarks{Problem \arabic{#1} (continued)}{Problem \arabic{#1} continued on next page\ldots}\nobreak{}
}

\newcommand{\exitProblemHeader}[1]{
    \nobreak\extramarks{Problem \arabic{#1} (continued)}{Problem \arabic{#1} continued on next page\ldots}\nobreak{}
    \stepcounter{#1}
    \nobreak\extramarks{Problem \arabic{#1}}{}\nobreak{}
}

\setcounter{secnumdepth}{0}
\newcounter{partCounter}
\newcounter{homeworkProblemCounter}
\setcounter{homeworkProblemCounter}{1}
\nobreak\extramarks{Problem \arabic{homeworkProblemCounter}}{}\nobreak{}

\newenvironment{homeworkProblem}{
    \section{Problem \arabic{homeworkProblemCounter}}
    \setcounter{partCounter}{1}
    \enterProblemHeader{homeworkProblemCounter}
}{
    \exitProblemHeader{homeworkProblemCounter}
}

\newcommand{\hmwkTitle}{Homework\ \#4}
\newcommand{\hmwkDueDate}{February 12, 2014}
\newcommand{\hmwkClass}{Stat330}
\newcommand{\hmwkClassTime}{Section A}
\newcommand{\hmwkClassInstructor}{Mr. Lanker}
\newcommand{\hmwkAuthorName}{Josh Davis}

\title{
    \vspace{2in}
    \textmd{\textbf{\hmwkClass:\ \hmwkTitle}}\\
    \normalsize\vspace{0.1in}\small{Due\ on\ \hmwkDueDate\ at 3:10pm}\\
    \vspace{0.1in}\large{\textit{\hmwkClassInstructor\ \hmwkClassTime}}
    \vspace{3in}
}

\author{\textbf{\hmwkAuthorName}}
\date{}

\newcommand{\alg}[1]{\textsc{\bfseries \footnotesize #1}}
\newcommand{\deriv}[1]{\frac{\mathrm{d}}{\mathrm{d}x} (#1)}
\newcommand{\dx}{\mathrm{d}x}
\newcommand{\solution}{\textbf{\large Solution}}

\renewcommand{\part}[1]{\textbf{\large Part \Alph{partCounter}}\stepcounter{partCounter}\\}

\newcommand{\E}{\mathrm{E}}
\newcommand{\Var}{\mathrm{Var}}
\newcommand{\Cov}{\mathrm{Cov}}
\newcommand{\dist}[1]{\sim \mathrm{#1}}

\begin{document}

\maketitle

\pagebreak

\begin{homeworkProblem}
    The \textit{discrete uniform} distribution defined on \([a,b]\), where
    \(a\) and \(b\) are integers, is the probability distribution where any
    number between \(a\) and \(b\) is equally likely to occur. Let \(X\) be the
    discrete uniform variable on \([a, b]\). The pmf of \(X\) is:

    \[
        p(x) = \left\{
            \begin{array}{ll}
                \frac{
                    1
                }{
                    b - a + 1
                }& \quad x \in \{a, a + 1, \ldots, b - 1, b\}
                \\
                0 & \quad \mbox{any other } x
            \end{array}
        \right.
    \]

    \part

    Find \(\E[X]\).
    \\

    \solution

    Solution.
    \\

    \part

    Suppose \(c\) is an integer between \(a\) and \(b\). How does \(p_x(c)\)
    change as \(a\) and \(b\) move further away from each other?
    \\

    \solution

    Solution.
\end{homeworkProblem}

\pagebreak

\begin{homeworkProblem}
    Considering the game of \textit{Monopoly}, answer the following questions.
    \\

    Let \(X\) be the outcome of a single roll of the dice, with ``success''
    considered to be rolling doubles and ``failure'' rolling anything else.
    Then \(X \dist{Bernoulli}(p)\).
    \\

    \part

    What is \(p\)?
    \\

    \solution

    Solution.
    \\

    \part

    What are \(\E[X]\) and \(\Var[X]\)?
    \\

    \solution

    Solution
    \\

    \part

    Construct a graph of the cumulative distribution function \(F_X (t)\).
    \\

    \solution

    Solution.
\end{homeworkProblem}

\pagebreak

\begin{homeworkProblem}
\end{homeworkProblem}

\pagebreak

\begin{homeworkProblem}
\end{homeworkProblem}

\pagebreak

\begin{homeworkProblem}
\end{homeworkProblem}

\pagebreak

\begin{homeworkProblem}
\end{homeworkProblem}

\pagebreak

\begin{homeworkProblem}
\end{homeworkProblem}

\pagebreak

\begin{homeworkProblem}
    The probability that you attend lecture is .95. If there are 30 lectures
    during the semester, what is the probability that you miss 2 or more
    lectures throughout the semester? There is no attendance requirement, and
    the probability of missing any lecture is independent of other lectures.
    \\

    \solution

    Solution.
\end{homeworkProblem}

\end{document}
