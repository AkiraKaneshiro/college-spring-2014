\documentclass{article}

\usepackage{fancyhdr}
\usepackage{extramarks}
\usepackage{amsmath}
\usepackage{amsthm}
\usepackage{amsfonts}

\topmargin=-0.45in
\evensidemargin=0in
\oddsidemargin=0in
\textwidth=6.5in
\textheight=9.0in
\headsep=0.25in

\linespread{1.1}

\pagestyle{fancy}
\lhead{\hmwkAuthorName}
\chead{\hmwkClass\ (\hmwkClassInstructor\ \hmwkClassTime): \hmwkTitle}
\rhead{\firstxmark}
\lfoot{\lastxmark}
\cfoot{\thepage}

\renewcommand\headrulewidth{0.4pt}
\renewcommand\footrulewidth{0.4pt}

\setlength\parindent{0pt}

\newcommand{\enterProblemHeader}[1]{
    \nobreak\extramarks{}{Problem \arabic{#1} continued on next page\ldots}\nobreak{}
    \nobreak\extramarks{Problem \arabic{#1} (continued)}{Problem \arabic{#1} continued on next page\ldots}\nobreak{}
}

\newcommand{\exitProblemHeader}[1]{
    \nobreak\extramarks{Problem \arabic{#1} (continued)}{Problem \arabic{#1} continued on next page\ldots}\nobreak{}
    \stepcounter{#1}
    \nobreak\extramarks{Problem \arabic{#1}}{}\nobreak{}
}

\setcounter{secnumdepth}{0}
\newcounter{partCounter}
\newcounter{homeworkProblemCounter}
\setcounter{homeworkProblemCounter}{1}
\nobreak\extramarks{Problem \arabic{homeworkProblemCounter}}{}\nobreak{}

\newenvironment{homeworkProblem}{
    \section{Problem \arabic{homeworkProblemCounter}}
    \setcounter{partCounter}{1}
    \enterProblemHeader{homeworkProblemCounter}
}{
    \exitProblemHeader{homeworkProblemCounter}
}

\newcommand{\hmwkTitle}{Homework\ \#4}
\newcommand{\hmwkDueDate}{February 12, 2014}
\newcommand{\hmwkClass}{Stat330}
\newcommand{\hmwkClassTime}{Section A}
\newcommand{\hmwkClassInstructor}{Mr. Lanker}
\newcommand{\hmwkAuthorName}{Josh Davis}

\title{
    \vspace{2in}
    \textmd{\textbf{\hmwkClass:\ \hmwkTitle}}\\
    \normalsize\vspace{0.1in}\small{Due\ on\ \hmwkDueDate\ at 3:10pm}\\
    \vspace{0.1in}\large{\textit{\hmwkClassInstructor\ \hmwkClassTime}}
    \vspace{3in}
}

\author{\textbf{\hmwkAuthorName}}
\date{}

\newcommand{\alg}[1]{\textsc{\bfseries \footnotesize #1}}
\newcommand{\deriv}[1]{\frac{\mathrm{d}}{\mathrm{d}x} (#1)}
\newcommand{\dx}{\mathrm{d}x}
\newcommand{\solution}{\textbf{\large Solution}}

\renewcommand{\part}[1]{\textbf{\large Part \Alph{partCounter}}\stepcounter{partCounter}\\}

\newcommand{\E}{\mathrm{E}}
\newcommand{\Var}{\mathrm{Var}}
\newcommand{\Cov}{\mathrm{Cov}}
\newcommand{\dist}[1]{\sim \mathrm{#1}}

\begin{document}

\maketitle

\pagebreak

\begin{homeworkProblem}
    The \textit{discrete uniform} distribution defined on \([a,b]\), where
    \(a\) and \(b\) are integers, is the probability distribution where any
    number between \(a\) and \(b\) is equally likely to occur. Let \(X\) be the
    discrete uniform variable on \([a, b]\). The pmf of \(X\) is:

    \[
        p_X(x) = \left\{
            \begin{array}{ll}
                \frac{
                    1
                }{
                    b - a + 1
                }& \quad x \in \{a, a + 1, \ldots, b - 1, b\}
                \\
                0 & \quad \mbox{any other } x
            \end{array}
        \right.
    \]

    \part

    Find \(\E[X]\).
    \\

    \solution

    The expected value is the sum of the probabilities multiplied by the \(x\)
    value:

    \[
        \begin{split}
            \E[X] &= \sum_{x = a}^{b} x \cdot p_X(x)
            \\
            &= \sum_{x = a}^{b} x \cdot \frac{1}{b - a + 1}
            \\
            &= \frac{1}{b - a + 1} \sum_{x = a}^{b} x
            \\
            &= \frac{1}{b - a + 1} \left[
                \sum_{x = a}^{b} x - \sum_{x = 1}^{a - 1} x
            \right]
            \\
            &= \frac{1}{b - a + 1} \left[
                \frac{b}{2}(b + 1) - \frac{a - 1}{2}(a - 1 + 1)
            \right]
            \\
            &= \frac{1}{b - a + 1} \cdot \frac{1}{2} \cdot \left[
                b^2 + b - a^2 + a
            \right]
            \\
            &= \frac{1}{b - a + 1} \cdot \frac{1}{2} \cdot \left[
                (b^2 - a^2) + (b + a)
            \right]
            \\
            &= \frac{1}{b - a + 1} \cdot \frac{1}{2} \cdot \left[
                (b + a)(b - a) + (b + a)
            \right]
            \\
            &= \frac{1}{b - a + 1} \cdot \frac{1}{2} \cdot \left[
                (b + a)(b - a + 1)
            \right]
            \\
            &= \frac{b + a}{2}
        \end{split}
    \]

    \part

    Suppose \(c\) is an integer between \(a\) and \(b\). How does \(p_X(c)\)
    change as \(a\) and \(b\) move further away from each other?
    \\

    \solution

    Since since it is a uniform distribution, the value of \(p_X(c)\) will be
    dependent on the distance of \(a\) and \(b\) which is given as \(\frac{1}{b
    - a + 1}\). Thus as \(a\) and \(b\) get farther from each other, that is
    making the denominator of the fraction bigger, thus the probability
    decreases.
    \\

    The highest the probability can be is when \(a = b\) which would make the
    fraction be \(1 / 1\). Thus, again, it makes sense that it decreases as
    \(a\) and \(b\) grow apart.
\end{homeworkProblem}

\pagebreak

\begin{homeworkProblem}
    Considering the game of \textit{Monopoly}, answer the following questions.
    \\

    Let \(X\) be the outcome of a single roll of the dice, with ``success''
    considered to be rolling doubles and ``failure'' rolling anything else.
    Then \(X \dist{Bernoulli}(p)\).
    \\

    \part

    What is \(p\)?
    \\

    \solution

    The probability of rolling doubles, or \(n/\left| \Omega \right|\) where
    the size of \(\Omega = 36\), the total possibilities and \(n = 6\) because
    a 1 can only match a 1, and a 2 can only match a 2, and so on. Thus \(p =
    6/36 = 1/6 \approx .167\).
    \\

    \part

    What are \(\E[X]\) and \(\Var[X]\)?
    \\

    \solution

    Since our distribution is a Bernoulli distribution, the expectation of
    such a distribution is \(\E[X] = p\), so with probability \(p \approx
    .167\), the expectation is thus \(\E[X] = p\).
    \\

    Similarly, the variance for a Bernoulli distribution is defined as
    \(\Var[X] = pq\) with \(p\) and \(q = 1 - p \approx .833\) thus \(\Var[X] =
    pq \approx .139\).
    \\

    \part

    Construct a graph of the cumulative distribution function \(F_X (t)\).
    \\

    \solution

    Solution.
    \\

    Now let's take a look at the number of turns that are needed until doubles
    are rolled.  Let \(Y\) be the random variable representing the number of
    rolls of the dice until doubles comes up. Then \(Y \dist{Geometric}(p)\).
    \\

    \part

    Using your answers for 2a for \(p\), what is the expected number of turns a
    player will need to get out of jail? (Hint: this is \(\E[Y]\)).
    \\

    \solution

    Solution.
    \\

    \part

    What is the probability that a player will need four or more rolls to get
    out of jail?
    \\

    \solution

    Solution
\end{homeworkProblem}

\pagebreak

\begin{homeworkProblem}
    A student is taking a 10 question multiple choice exam, where each question
    has four possible answers and the correct answer is chosen randomly and
    independent of all other answers. If the student guesses ``C'' on every
    question, then \(X\), the number of questions that the student answers
    correctly, follows a \(\dist{Binomial}(10, 0.25)\) distribution.
    \\

    \part

    Find \(\E[X]\) and \(\Var[X]\).
    \\

    \solution

    Since the distribution is a Binomial one, the expected value of it is
    \(\E[X] = np\) where \(n\) is the number of trials and \(p\) is the
    probability of a success. Thus \(\E[X] = (10)(.25) = 2.5\).
    \\

    For the variance of a Binomial distribution, it is defined as \(\Var[X] =
    npq\) where \(n\) and \(p\) are the same as previously, and \(q = 1 - p =
    .75\). Thus \(\Var[X] = (10)(.25)(.75) = 1.875\).
    \\

    \part

    Does \(\E[X]\) represent a possible outcome? Does this matter? Why or why
    not?
    \\

    \solution

    No, it isn't possible to get 2.5 questions right on the exam. This doesn't
    matter because the expectation is how many someone would expect to get
    right on 10 questions for guessing C over and over again for a very long
    time. It isn't the number he \textbf{will} get right, but rather the number
    he \textbf{should} get right. Thus it makes more sense that over a 20
    question exam, the guesser would be expected to get 5 right.
    \\

    \part

    What is the probability that the student answers no questions correctly?
    \\

    \solution

    In this case, \(n = 10\) questions and given our values of \(p = .25\) and \(q = .75\).
    We want to see the probability of him getting 0 questions right. Thus the value
    we are looking for is when \(x = 0\) because \(x\). Thus given our pmf,
    \(p(x) = \binom{n}{x}p^x q^{n - x}\). Inserting our values gives us:

    \[
        \begin{split}
            P[X = 0] = p(0) &= \binom{10}{0} p^0 q^{10 - 0}
            \\
            &= \frac{10!}{0! 10!} 1 (.75)^10
            \\
            &= 1 \cdot 1 \cdot .75^10
            \\
            &\approx .056
        \end{split}
    \]

    \part

    The student will pass the exam if he answers 6 or more questions correctly.
    What is the probability that the student passes the exam?
    \\

    \solution

    Using the above equation, we want to see the probability of \(P[X \geq 6]\).
    Which gives us the following:

    \[
        \begin{split}
            P[X \geq 6] &= p(6) + p(7) + p(8) + p(9) + p(10)
            \\
            &= \sum_{x = 6}^{10} \binom{10}{x} p^x q^{10 - x}
            \\
            &= 1 - P[X < 6]
            \\
            &= 1 - P[X \leq 5] \quad \mbox{because its discrete}
            \\
            &\approx 1 - .980 \quad \mbox{using the cdf function, } F(x)
            \\
            &\approx .020
        \end{split}
    \]
\end{homeworkProblem}

\pagebreak

\begin{homeworkProblem}
    In some city, the probability of rain on any day is 0.60, determined using
    historical climate records. It is known that on rainy days the number of
    traffics has a \(\dist{Poisson}(4)\) distribution.
    \\

    An accident occurred years ago and there are no precise weather records
    available, but we do know that on the day of the accident in question there
    were 8 accidents in the city.  Use Baye's Rule to determine the probability
    that it was in fact a rainy day.
    \\

    \solution

    Solution.

\end{homeworkProblem}

\pagebreak

\begin{homeworkProblem}
    Fred Hoiberg had a free throw shooting percentage of 85.4\%. Suppose that in
    pratice one day he shoots free throws until his first miss. Let \(X\) be
    the number of shots he takes.
    \\

    \part

    Define the probability distribution of \(X\), carefully determining the
    appropriate parameter(s).
    \\

    \solution

    The probability distribution that this problem will take is the Geometric
    distribution on the random variable \(X\) where \(X\) is the number of
    shots.  The parameters just need to be \(x\), the number of shots until a
    success and \(p\), the probability of making a shot which was given, thus
    \(p = .854\).
    \\

    The equation for the probability distribution is then \(P(x) = (1 - p)^{x -
    1} p\).
    \\

    \part

    What is the probability that Hoiberg makes his first two shots but misses
    his third short?
    \\

    \solution

    The probability of two successes and a failure would be:
    \[
        p \cdot p \cdot (1 - p) = (.854) (.854) (.146) = .106
    \]

    \part

    What is the probability that Hoiberg makes more than 15 shots?
    \\

    \solution

    Solution.
    \\

    \part

    What is the probability that Hoiberg takes between two and four (inclusive)
    shots?
    \\

    \solution

    Solution.
\end{homeworkProblem}

\pagebreak

\begin{homeworkProblem}
    On average, a certain region in Alaska experiences on average 12 severe
    earthquakes every 10 years. Assume that the timings of sever earthquakes
    are independent. If you have a three year assignment in this region of
    Alaska:
    \\

    \part

    What is the probability that there are no severe earthquakes during your
    assignment?
    \\

    \solution

    This problem takes the form of the Poisson distribution. We have a rate
    and a value that has one of the units of the rate. Thus we can calculate
    our lambda, \(\lambda = (12 / 10) (3) = 3.6\) occurences (earthquakes).
    \\

    We know the pmf of a Poisson distribution is: \(P(x) =
    e^{-\lambda}\frac{\lambda^x}{x!}\). We want to find the probability that
    there are no severe earthquakes or \(P[X = 0]\). Using the above equation,
    \(P[X = 0] = .027\).
    \\

    \part

    What is the probability that there are three or more severe earthquakes
    during your assignment?
    \\

    \solution

    Using the above information, we want to know \(P[X \geq 3]\) so
    we can calculate it as follows:

    \[
        \begin{split}
            P[X \geq 3]
            &= 1 - P[X < 3]
            \\
            &= 1 - P[X = 0] - P[X = 1] - P[X = 2]
            \\
            &\approx 1 - .027 - .098 - .177 \quad \mbox{using the above formula}
            \\
            &\approx .698
        \end{split}
    \]
\end{homeworkProblem}

\begin{homeworkProblem}
    A type of photo paper has on average 1.5 blemishes per square foot. You can
    reasonably assume that blemishes are randomly distributed. If the
    photograph you are printing is .4 square feet, what is the probability that
    the randomly selected photo paper for your photograph will have at least
    one blemish?
    \\

    \solution

    This problem takes the form of the Poisson distribution. We have a rate
    and a value that has one of the units of the rate. Thus we can calculate
    our lambda, \(\lambda = 1.5 * 0.4 = .6\) occurences (blemishes).
    \\

    We know the pmf of a Poisson distribution is: \(P(x) =
    e^{-\lambda}\frac{\lambda^x}{x!}\). We want to find the probability that we
    get at least one blemish which is \(P[X \geq 1]\). Then:

    \[
        \begin{split}
            P[X \geq 1] &= 1 - P[X < 1]
            \\
            &= 1 - P[X = 0]
            \\
            &\approx 1 - .549 \quad \mbox{by using the equation above}
            \\
            &\approx .451
        \end{split}
    \]

\end{homeworkProblem}

\begin{homeworkProblem}
    The probability that you attend lecture is .95. If there are 30 lectures
    during the semester, what is the probability that you miss 2 or more
    lectures throughout the semester? There is no attendance requirement, and
    the probability of missing any lecture is independent of other lectures.
    \\

    \solution

    If we think about attend as a success and missing lecture as a failure,
    then this takes the form of a Binomial distribution.
    \\

    This gives us the equation of \(p(x) = \binom{30}{x} p^x q^{30 - x}\)
    where \(p = .95\) and thus \(q = .05\).
    \\

    The probability of missing 2 or more lectures is the same as attending
    less than 29 lectures. Thus we can write it as \(P[X \leq 28]\). This can
    also be written as: \(1 - P[X > 28]\):

    \[
        \begin{split}
            P[X \leq 28] &= 1 - P[X > 28]
            \\
            &= 1 - p(29) - p(30)
            \\
            &\approx 1 - .339 - .215 \quad \mbox{by using the equation above}
            \\
            &\approx .446
        \end{split}
    \]
\end{homeworkProblem}

\end{document}
