\documentclass{article}

\usepackage{fancyhdr}
\usepackage{extramarks}
\usepackage{amsmath}
\usepackage{amsthm}
\usepackage{amsfonts}

\topmargin=-0.45in
\evensidemargin=0in
\oddsidemargin=0in
\textwidth=6.5in
\textheight=9.0in
\headsep=0.25in

\linespread{1.1}

\pagestyle{fancy}
\lhead{\hmwkAuthorName}
\chead{\hmwkClass\ (\hmwkClassInstructor\ \hmwkClassTime): \hmwkTitle}
\rhead{\firstxmark}
\lfoot{\lastxmark}
\cfoot{\thepage}

\renewcommand\headrulewidth{0.4pt}
\renewcommand\footrulewidth{0.4pt}

\setlength\parindent{0pt}

\newcommand{\enterProblemHeader}[1]{
    \nobreak\extramarks{}{Problem \arabic{#1} continued on next page\ldots}\nobreak{}
    \nobreak\extramarks{Problem \arabic{#1} (continued)}{Problem \arabic{#1} continued on next page\ldots}\nobreak{}
}

\newcommand{\exitProblemHeader}[1]{
    \nobreak\extramarks{Problem \arabic{#1} (continued)}{Problem \arabic{#1} continued on next page\ldots}\nobreak{}
    \stepcounter{#1}
    \nobreak\extramarks{Problem \arabic{#1}}{}\nobreak{}
}

\setcounter{secnumdepth}{0}
\newcounter{partCounter}
\newcounter{homeworkProblemCounter}
\setcounter{homeworkProblemCounter}{1}
\nobreak\extramarks{Problem \arabic{homeworkProblemCounter}}{}\nobreak{}

\newenvironment{homeworkProblem}{
    \section{Problem \arabic{homeworkProblemCounter}}
    \setcounter{partCounter}{1}
    \enterProblemHeader{homeworkProblemCounter}
}{
    \exitProblemHeader{homeworkProblemCounter}
}

\newcommand{\hmwkTitle}{Homework\ \#4}
\newcommand{\hmwkDueDate}{February 12, 2014}
\newcommand{\hmwkClass}{Stat330}
\newcommand{\hmwkClassTime}{Section A}
\newcommand{\hmwkClassInstructor}{Mr. Lanker}
\newcommand{\hmwkAuthorName}{Josh Davis}

\title{
    \vspace{2in}
    \textmd{\textbf{\hmwkClass:\ \hmwkTitle}}\\
    \normalsize\vspace{0.1in}\small{Due\ on\ \hmwkDueDate\ at 3:10pm}\\
    \vspace{0.1in}\large{\textit{\hmwkClassInstructor\ \hmwkClassTime}}
    \vspace{3in}
}

\author{\textbf{\hmwkAuthorName}}
\date{}

\newcommand{\alg}[1]{\textsc{\bfseries \footnotesize #1}}
\newcommand{\deriv}[1]{\frac{\mathrm{d}}{\mathrm{d}x} (#1)}
\newcommand{\dx}{\mathrm{d}x}
\newcommand{\solution}{\textbf{\large Solution}}

\renewcommand{\part}[1]{\textbf{\large Part \Alph{partCounter}}\stepcounter{partCounter}\\}

\newcommand{\E}{\mathrm{E}}
\newcommand{\Var}{\mathrm{Var}}
\newcommand{\Cov}{\mathrm{Cov}}
\newcommand{\dist}[1]{\sim \mathrm{#1}}

\begin{document}

\maketitle

\pagebreak

\begin{homeworkProblem}
    The \textit{discrete uniform} distribution defined on \([a,b]\), where
    \(a\) and \(b\) are integers, is the probability distribution where any
    number between \(a\) and \(b\) is equally likely to occur. Let \(X\) be the
    discrete uniform variable on \([a, b]\). The pmf of \(X\) is:

    \[
        p_X(x) = \left\{
            \begin{array}{ll}
                \frac{
                    1
                }{
                    b - a + 1
                }& \quad x \in \{a, a + 1, \ldots, b - 1, b\}
                \\
                0 & \quad \mbox{any other } x
            \end{array}
        \right.
    \]

    \part

    Find \(\E[X]\).
    \\

    \solution

    The expected value is the sum of the probabilities multiplied by the \(x\)
    value:

    \[
        \begin{split}
            \E[X] &= \sum_{x = a}^{b} x \cdot p_X(x)
            \\
            &= \sum_{x = a}^{b} x \cdot \frac{1}{b - a + 1}
            \\
            &= \frac{1}{b - a + 1} \sum_{x = a}^{b} x
            \\
            &= \frac{1}{b - a + 1} \left[
                \sum_{x = a}^{b} x - \sum_{x = 1}^{a - 1} x
            \right]
            \\
            &= \frac{1}{b - a + 1} \left[
                \frac{b}{2}(b + 1) - \frac{a - 1}{2}(a - 1 + 1)
            \right]
            \\
            &= \frac{1}{b - a + 1} \cdot \frac{1}{2} \cdot \left[
                b^2 + b - a^2 + a
            \right]
            \\
            &= \frac{1}{b - a + 1} \cdot \frac{1}{2} \cdot \left[
                (b^2 - a^2) + (b + a)
            \right]
            \\
            &= \frac{1}{b - a + 1} \cdot \frac{1}{2} \cdot \left[
                (b + a)(b - a) + (b + a)
            \right]
            \\
            &= \frac{1}{b - a + 1} \cdot \frac{1}{2} \cdot \left[
                (b + a)(b - a + 1)
            \right]
            \\
            &= \frac{b + a}{2}
        \end{split}
    \]

    \part

    Suppose \(c\) is an integer between \(a\) and \(b\). How does \(p_X(c)\)
    change as \(a\) and \(b\) move further away from each other?
    \\

    \solution

    Since since it is a uniform distribution, the value of \(p_X(c)\) will be
    dependent on the distance of \(a\) and \(b\) which is given as \(\frac{1}{b
    - a + 1}\). Thus as \(a\) and \(b\) get farther from each other, that is
    making the denominator of the fraction bigger, thus the probability
    decreases.
    \\

    The highest the probability can be is when \(a = b\) which would make the
    fraction be \(1 / 1\). Thus, again, it makes sense that it decreases as
    \(a\) and \(b\) grow apart.
\end{homeworkProblem}

\pagebreak

\begin{homeworkProblem}
    Considering the game of \textit{Monopoly}, answer the following questions.
    \\

    Let \(X\) be the outcome of a single roll of the dice, with ``success''
    considered to be rolling doubles and ``failure'' rolling anything else.
    Then \(X \dist{Bernoulli}(p)\).
    \\

    \part

    What is \(p\)?
    \\

    \solution

    Solution.
    \\

    \part

    What are \(\E[X]\) and \(\Var[X]\)?
    \\

    \solution

    Solution
    \\

    \part

    Construct a graph of the cumulative distribution function \(F_X (t)\).
    \\

    \solution

    Solution.
    \\

    Now let's take a look at the number of turns that are needed until doubles are rolled.
    Let \(Y\) be the random variable representing the number of rolls of the dice
    until doubles comes up. Then \(Y \dist{Geometric}(p)\).
    \\

    \part

    Using your answers for 2a for \(p\), what ist he expected number of turns a player will
    need to get out of jail? (Hint: this is \(\E[Y]\)).
    \\

    \solution

    Solution.
    \\

    \part

    What is the probability that a player will need four or more rolls to get
    out of jail?
    \\

    \solution

    Solution
\end{homeworkProblem}

\pagebreak

\begin{homeworkProblem}
    A student is taking a 10 question multiple choice exam, where each question has four possible
    answers and the correct answer is chosen randomly and independent of all other answers. If the
    student guesses ``C'' on every question, then \(X\), the number of questions that the student
    answers correctly, follows a \(\dist{Binomial}(10, 0.25)\) distribution.
    \\

    \part

    Find \(\E[X]\) and \(\Var[X]\).
    \\

    \solution

    Solution.
    \\

    \part

    Does \(\E[X]\) represent a possible outcome? Does this matter? Why or why
    not?
    \\

    \solution

    Solution.
    \\

    \part

    What is the probability that the student answers no questions correctly?
    \\

    \solution

    Solution.
    \\

    \part

    The student will pass the exam if he answers 6 or more questions correctly.
    What is the probability that the student passes the exam?
    \\

    \solution

    Solution.
\end{homeworkProblem}

\pagebreak

\begin{homeworkProblem}
    In some city, the probability of rain on any day is 0.60, determined using
    historical climate records. It is known that on rainy days the number of
    traffics has a \(\dist{Poisson}(4)\) distribution.
    \\

    An accident occurred years ago and there are no prcise weather records
    available, but we do know that on the day of the accident in question there
    were 8 accidents in the city.  Use Baye's Rule to determine the probability
    that it was in fact a rainy day.
    \\

    \solution

    Solution.

\end{homeworkProblem}

\pagebreak

\begin{homeworkProblem}
    Fred Hoiberg had a free throw shooting percentage of 85.4%. Suppose that in
    pratice one day he shoots free throws utnil his first miss. Let \(X\) be
    the number of shots he takes.
    \\

    \part

    Define the probability distribution of \(X\), carefully determining the
    appropriate parameter(s).
    \\

    \solution

    Solution.
    \\

    \part

    What is the probability that Hoiberg makes his first two shorts but misses
    his third short?
    \\

    \solution

    Solution.
    \\

    \part

    What is the probability that Hoiberg makes more than 15 shots?
    \\

    \solution

    Solution.
    \\

    \part

    What is the probability that Hoiberg takes between two and four (inclusive)
    shots?
    \\

    \solution

    Solution.
\end{homeworkProblem}

\pagebreak

\begin{homeworkProblem}
    On average, a certain region in Alaska experiences on average 12 severe
    earthquakes every 10 years. Assume that the timings of sever earthquakes
    are independent. If you have a three year assignment in this region of
    Alaska:
    \\

    \part

    What is the probability that there are no severe earthquakes during your
    assignment?
    \\

    \solution

    Solution.
    \\

    \part

    What is the probability that there are three or more severe earthquakes
    during your assignment?
    \\

    \solution

    Solution.
\end{homeworkProblem}

\pagebreak

\begin{homeworkProblem}
    A type of photo paper has on average 1.5 blemishes per square foot. You can
    reasonably assume that blemishes are randomly distributed. If the
    photograph you are printing is .4 square feet, what is the probability that
    the randomly selected photo paper for your photograph will have at least
    one blemish?
    \\

    \solution

    Solution.
\end{homeworkProblem}

\pagebreak

\begin{homeworkProblem}
    The probability that you attend lecture is .95. If there are 30 lectures
    during the semester, what is the probability that you miss 2 or more
    lectures throughout the semester? There is no attendance requirement, and
    the probability of missing any lecture is independent of other lectures.
    \\

    \solution

    Solution.
\end{homeworkProblem}

\end{document}
