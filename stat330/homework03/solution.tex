\documentclass{article}

\usepackage{fancyhdr}
\usepackage{lastpage}
\usepackage{extramarks}
\usepackage[usenames,dvipsnames]{color}
\usepackage{amsmath}
\usepackage{amsthm}
\usepackage{amsfonts}
\usepackage{changepage}
\usepackage{lineno}
\usepackage[plain]{algorithm}
\usepackage{algpseudocode}
\usepackage{hyperref}
\usepackage{tikz}
\usepackage{pgfplots}

\usetikzlibrary{calc}

\topmargin=-0.45in
\evensidemargin=0in
\oddsidemargin=0in
\textwidth=6.5in
\textheight=9.0in
\headsep=0.25in

\linespread{1.1}

\pagestyle{fancy}
\lhead{\hmwkAuthorName}
\chead{\hmwkClass\ (\hmwkClassInstructor\ \hmwkClassTime): \hmwkTitle}
\rhead{\firstxmark}
\lfoot{\lastxmark}
\cfoot{}

\renewcommand\headrulewidth{0.4pt}
\renewcommand\footrulewidth{0.4pt}

\setlength{\floatsep}{100pt}
\renewcommand{\algorithmicrequire}{\textbf{Input:}}
\renewcommand{\algorithmicensure}{\textbf{Output:}}
\algrenewcomment[1]{\hfill // #1}

\setlength\parindent{0pt}

\hypersetup{colorlinks=true}

\newcommand{\enterProblemHeader}[1]{
    \nobreak\extramarks{}{Problem \arabic{#1} continued on next page\ldots}\nobreak{}
    \nobreak\extramarks{Problem \arabic{#1} (continued)}{Problem \arabic{#1} continued on next page\ldots}\nobreak{}
}

\newcommand{\exitProblemHeader}[1]{
    \nobreak\extramarks{Problem \arabic{#1} (continued)}{Problem \arabic{#1} continued on next page\ldots}\nobreak{}
    \stepcounter{#1}
    \nobreak\extramarks{Problem \arabic{#1}}{}\nobreak{}
}

\setcounter{secnumdepth}{0}
\newcounter{partCounter}
\newcounter{homeworkProblemCounter}
\setcounter{homeworkProblemCounter}{1}
\nobreak\extramarks{Problem \arabic{homeworkProblemCounter}}{}\nobreak{}

\newenvironment{homeworkProblem}{
    \section{Problem \arabic{homeworkProblemCounter}}
    \setcounter{partCounter}{1}
    \enterProblemHeader{homeworkProblemCounter}
}{
    \exitProblemHeader{homeworkProblemCounter}
}

\newcommand{\hmwkTitle}{Homework\ \#3}
\newcommand{\hmwkDueDate}{February 5, 2014}
\newcommand{\hmwkClass}{Stat330}
\newcommand{\hmwkClassTime}{Section A}
\newcommand{\hmwkClassInstructor}{Mr. Lanker}
\newcommand{\hmwkAuthorName}{Josh Davis}

\title{
    \vspace{2in}
    \textmd{\textbf{\hmwkClass:\ \hmwkTitle}}\\
    \normalsize\vspace{0.1in}\small{Due\ on\ \hmwkDueDate\ at 3:10pm}\\
    \vspace{0.1in}\large{\textit{\hmwkClassInstructor\ \hmwkClassTime}}
    \vspace{3in}
}

\author{\textbf{\hmwkAuthorName}}
\date{}

\newcommand{\alg}[1]{\textsc{\bfseries \footnotesize #1}}
\newcommand{\deriv}[1]{\frac{\mathrm{d}}{\mathrm{d}x} (#1)}
\newcommand{\dx}{\mathrm{d}x}
\newcommand{\solution}{\textbf{\large Solution}}

\renewcommand{\part}[1]{\textbf{\large Part \Alph{partCounter}}\stepcounter{partCounter}\\}

\newcommand{\E}{\mathrm{E}}
\newcommand{\Var}{\mathrm{Var}}
\newcommand{\Cov}{\mathrm{Cov}}

\pgfmathdeclarefunction{problem3}{1}{%
  \pgfmathparse{%
    (and(   1,    #1<0)*(0)            +%
    (and(#1>= 1,  #1< 2)*(0.1)          +%
    (and(#1>= 2,  #1< 3)*(0.3)          +%
    (and(#1>= 3,  #1< 4)*(0.6) +%
    (and(#1>= 4,    1  )*(1)%
    }%
}

\begin{document}

\maketitle

\pagebreak

\begin{homeworkProblem}
    Define a discrete random variable and the sample space.
    \\

    \part

    The number of bowling games needed for you to have at least 100 points.
    \\

    \solution

    \(X =\) number of bowling games to score at least 100 points. \(\Omega =
    \{1, 2, 3, 4, ...\}\).
    \\

    \part

    Analyze how many accidents occur at the intersection of Lincoln Way \&
    Welch Ave. during any week.
    \\

    \solution

    \(X =\) number of accidents at Lincoln Way \& Welch during a week. \(\Omega
    = \{0, 1, 2, 3, 4, ...\}\).
    \\

    \part

    You play a game where you roll a 6-sided die and win a number of points
    equal to 3 divided by your roll.
    \\

    \solution

    \(X =\) number when dividing 3 by the number when rolling a 6-sided die.
    \(\Omega = \{3, \frac{3}{2}, 1, \frac{3}{4}, \frac{3}{5}, \frac{3}{6}\}\).
    \\
\end{homeworkProblem}

\pagebreak

\begin{homeworkProblem}
    Five balls numbered 1, 3, 5, 7 and 9 are placed in an urn. Two balls are
    randomly selected from the five (without replacement), and their numbers
    noted. Find the probability distribution for the following.
    \\

    \part

    The \textit{largest} of the two sampled numbers.
    \\

    \solution

    Solution.
    \\

    \part

    The \textit{average} of the two sampled numbers.
    \\

    \solution

    Solution.
\end{homeworkProblem}

\pagebreak

\begin{homeworkProblem}
    Calculate the cumulative probability function for the valid probability
    mass function:

    \[
        P(x) = \left\{
            \begin{array}{ll}
                x/10 & \quad x = 1,2,3,4
                \\
                0 & \quad \mbox{any other } x
            \end{array}
        \right.
    \]

    Carefully plot the cumulative probability function for \(X\), \(F(x)\),
    labeling all axes.
    \\

    \solution

    \[
        F(x) = \left\{
            \begin{array}{ll}
                0 & \quad x < 1
                \\
                1/10 & \quad 1 \leq x < 2
                \\
                3/10 & \quad 2 \leq x < 3
                \\
                6/10 & \quad 3 \leq x < 4
                \\
                1 & \quad x = 4
                \\
                1 & \quad x > 4
            \end{array}
        \right.
    \]

    \begin{figure}[!h]
        \centering
        \begin{tikzpicture}
            \begin{axis}
                \foreach \xStart/\xEnd  in {0/1, 1/2, 2/3, 3/4, 4/5} {
                    \addplot[domain=\xStart:\xEnd, blue, samples=10, ultra thick] {problem3(x)};
                }

                \draw [draw=blue, fill=white, thick] (axis cs: 1, 0) circle (2.0pt);
                \draw [draw=blue, fill=blue,  thick] (axis cs: 1, 0.1) circle (2.0pt);

                \draw [draw=blue, fill=white, thick] (axis cs: 2, .1) circle (2.0pt);
                \draw [draw=blue, fill=blue,  thick] (axis cs: 2, 0.3) circle (2.0pt);

                \draw [draw=blue, fill=white, thick] (axis cs: 3, .3) circle (2.0pt);
                \draw [draw=blue, fill=blue,  thick] (axis cs: 3, 0.6) circle (2.0pt);

                \draw [draw=blue, fill=white, thick] (axis cs: 4, .6) circle (2.0pt);
                \draw [draw=blue, fill=blue,  thick] (axis cs: 4, 1) circle (2.0pt);
            \end{axis}
        \end{tikzpicture}
    \end{figure}
\end{homeworkProblem}

\pagebreak

\begin{homeworkProblem}
    Every day, the number of network blackouts has a distribution (pmf):

    \begin{table}[ht]
        \centering
        \begin{tabular}{c || c | c | c}
            \(x\) & 0 & 1 & 2
            \\
            \hline
            \(P(x)\) & .7 & .2 & .1
        \end{tabular}
    \end{table}

    A small internet trading company estimates that each network balckout
    results in a \$500 loss. Compute expectation and variance of this company's
    daily loss due to blackouts.
    \\

    \solution

    Expectation:
    \[
        \begin{split}
            \E [X] &= \sum_0^2 xP(x)
            \\
            &= (0)(.7) + (1)(.2) + (2)(.1)
            \\
            &= 0 + .2 + .2
            \\
            &= .4
        \end{split}
    \]

    Variance:
    \[
        \begin{split}
            \Var [X] &= \E X^2 - {(\E X)}^2
            \\
            &= (0^2)(.7) + (1^2)(.2) + (2^2)(.1) - .4^2
            \\
            &= .4 + .2 - .16
            \\
            &= .44
        \end{split}
    \]
\end{homeworkProblem}

\pagebreak

\begin{homeworkProblem}
    Calculate the mean, variance, and standard deviation of the discrete
    probability distribution in question 3.
    \\

    \solution

    Solution.
\end{homeworkProblem}

\pagebreak

\begin{homeworkProblem}
    A single fair die is tossed once. Let \(Y\) be the number facing up. Find
    the expected value and variance of \(Y\).
    \\

    \solution

    Expectation:
    \[
        \begin{split}
            \E [X] &= \sum_0^2 xP(x)
            \\
            &= (1)(1/6) + (2)(1/6) + (3)(1/6) + (4)(1/6) + (5)(1/6) + (6)(1/6)
            \\
            &= 3.5
        \end{split}
    \]

    Variance:
    \[
        \begin{split}
            \Var [X] &= \E X^2 - {(\E X)}^2
            \\
            &= (1^2)(1/6) + (2^2)(1/6) + (3^2)(1/6) + (4^2)(1/6) + (5^2)(1/6) + (6^2)(1/6) - (3.5)^2
            \\
            &= 2.92
        \end{split}
    \]
\end{homeworkProblem}

\end{document}
